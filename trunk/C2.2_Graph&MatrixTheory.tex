
\subsection{Graph Theory and Matrix Theory}

In this section, some basic concepts of the graph theory and matrix
theory will be introduced. They are used in the analysis of convergence
or performance of consensus algorithm. Because consensus algorithm
actually relates to a matrix iteration algorithm, it is necessary
to introduce some of these theorems. For full information about matrix
theory, see \cite{Varga2010}, and the work \cite{Russell1994} states
more details about Laplacian matrix. However, some useful properties
of Laplacian matrix will to be introduced here. 

Let ${\cal G}=\left({\cal V},{\cal E},{\cal A}\right)$ be a graph
with $n$ nodes. The in-degree and out-degree of node $i$ are defined
by:

\begin{equation}
D_{in}\left(i\right)=\sum_{i=1}^{n}a_{i,j}
\end{equation}
\begin{equation}
D_{out}\left(i\right)=\sum_{j=1}^{n}a_{i,j}
\end{equation}
where $a_{i,j}$ is the elements of matrix ${\cal A}$. This definition
states that in-degree of node $i$ is the $i^{th}$ column sum of
matrix ${\cal A}$ and the out-degree of node $i$ is the $i^{th}$
row sum of matrix ${\cal A}.$ And the graph Laplacian matrix ${\cal L}$
induced by the ${\cal G}$ is the same as defined before, see \ref{eq:Graph Laplacian def.}.
In addition, we can find the relationship of ${\cal L}$ and ${\cal A}$.
\begin{equation}
{\cal L}=\Delta-{\cal A}
\end{equation}
 where $\Delta$ is a diagonal matrix $\Delta=\left[\Delta_{i,j}\right]$,
$\Delta_{i,j}=0$ for all $i\neq j$ and $\Delta_{ii}=D_{out}\left(i\right)$.

Note that we assume the diagonal elements $a_{i,i}$ of matrix ${\cal A}$
equal to zero for all $i$. Thus, the Laplacian matrix is only dependent
on the off-diagonal elements of ${\cal A}$. Moreover, if we assume
matrix ${\cal A}$ is non-negative, we can benefit from the properties
of non-negative matrix, and use them in optimization of convergence
rate of consensus algorithm.


\subsubsection{Irreducibility and Strong Connected Graph.}

For an undirected graph, the consensus $\mathbf{x}^{*}$ can be achieved
if and only if the graph is connected. (Note that the consensus is
a stable state of the system dynamic. For the average consensus problem
the consensus state is a state where all the network node converge
to the global average.) But for the directed graph, this condition
of achieving consensus becomes into if and only if the graph is strongly
connected. 

A directed graph is strongly connected if and only if for any two
distinct nodes $i$ and $j$, there exists a path that follows the
direction of the edges and connects $i$ and $j$ on the digraph. 
\begin{defn}
\label{Def.Irreducebility}for $n>1$, an $n\times n$ matrix $A\in R^{n\times n}$
is reducible if there exists an $n\times n$ permutation matrix $P$
such that $PAP^{T}$ is in block upper triangular form. 
\begin{equation}
PAP^{T}=\left[\begin{array}{cc}
A_{1,1} & A_{1,2,}\\
O & A_{2,2}
\end{array}\right]
\end{equation}
where $A_{1,1}$ is an $r\times r$ submatrix and $A_{2,2}$ is an
$\left(n-r\right)\times\left(n-r\right)$ submatrix, and $O$ is an
null matrix, $1\leq r<n$. If no such a permutation matrix exists,
the matrix $A$ is irreducible. If $n=1$, then $A$ is reducible
if $A=0$, and irreducible otherwise. 
\end{defn}
The relationship of the irreducible property of matrix $A$ and the
strong connected property of directed graph ${\cal G}\left(A\right)$
is stated by the following theorem.
\begin{thm}
An $n\times n$ complex matrix $A\in C^{n\times n}$ is irreducible
if and only if its directed graph ${\cal G}\left(A\right)$ is strongly
connected. \cite{Varga2010}
\end{thm}
The proof of this theorem is obvious. If a graph is strongly connected,
all the off diagonal elements of graph matrix $A$ cannot be vanished
by matrix permutation. Therefore, matrix $A$ doesn't exists the block
upper triangular form as given in Def. \ref{Def.Irreducebility}. 


\subsubsection{Spectral radius of a matrix}

(todo, delete this
\begin{enumerate}
\item bounds for the spectral radius: 
\item and Gerschgorin's theorem: 1.5 page 1.6; 

\begin{enumerate}
\item theorem 1.5; Corollary 1; Corollary 2; give me the possibility of
a sub-optimal distributed method. Page 16
\item Perron-Frobenius theorem: 2.1 page 30.
\item Lemma 2.5 on page 31 \cite{Varga2010}
\end{enumerate}
\end{enumerate}
todo delete)

Spectral radius of a matrix is one of the basic concepts in the matrix
iteration theory. It is defined by the largest eigenvalue of the matrix.
The matrix iteration is very useful in many applications, denoted
by $A,A^{2},A^{3},\ldots$. The power sequence is said to be convergent,
if and only if $\lim_{k\to\infty}A^{k}=O$, where $O$ is a zero matrix
with all zero entries. The following theorem states that the convergent
property is strongly connected with the spectral radius. 
\begin{thm}
\label{thm:Convergent <=00003D> p(A)<1} if $A\in C^{n\times n}$
is an $n\times n$ complex matrix, then $A$ is convergent if and
only if $\rho\left(A\right)<1.$\end{thm}
\begin{proof}
The proof uses the Jordan form of a matrix. For any matrix $A\in C^{n\times n}$,
there exists a nonsingular $n\times n$ matrix $T$, such that $A$
reduces into the Jordan normal form 
\begin{equation}
T^{-1}AT=J=\left[\begin{array}{cccc}
J_{1} &  &  & O\\
 & J_{2}\\
O &  & \ddots & J_{m}
\end{array}\right]
\end{equation}
 where each Jordan block $J_{i}$ is an $r_{i}\times r_{i}$ submatrix
in the form 
\begin{equation}
J_{i}=\left[\begin{array}{ccccc}
\lambda_{i} & 1\\
 & \lambda_{i} & 1\\
 &  & \lambda_{i} & \ddots\\
 &  &  & \ddots & 1\\
 &  &  &  & \lambda_{i}
\end{array}\right]
\end{equation}
Thus, the matrix $J$ and $A$ are similar and have the same eigenvalues
$\lambda_{i},i=1,\ldots m$

A direct computation of the power iteration of matrix $A$ will give
us the following equation
\[
A^{k}=TJ^{k}T^{-1}=T\left[\begin{array}{cccc}
J_{1}^{k} &  &  & O\\
 & J_{2}^{k}\\
O &  & \ddots & J_{m}^{k}
\end{array}\right]T^{-1}
\]
because the property of Jordan block, the power of each Jordan block
will have the form
\[
J_{i}^{2}=\left[\begin{array}{ccccc}
\lambda_{i}^{2} & 2\lambda_{i} & 1\\
 & \lambda_{i}^{2} & 2\lambda_{i} & \ddots\\
 &  & \lambda_{i}^{2} & \ddots & 1\\
 &  &  & \ddots & 2\lambda_{i}\\
 &  &  &  & \lambda_{i}^{2}
\end{array}\right],\mbox{if }r_{i}\geq3
\]
 and more generally, let $J_{i}^{k}=\left[c_{m,n}\left(i,k\right)\right]$,
$1\leq m$, $n\leq r_{i}$, and it has 
\[
c_{m,n}\left(i,k\right)=\begin{cases}
0 & n<m\\
\left(\begin{array}{c}
k\\
n-m
\end{array}\right)\lambda_{i}^{k-n+m} & m\leq n\leq\min\left(r_{i},k+m\right)\\
0 & k+m<n<r_{i}
\end{cases}
\]
 Since $\rho\left(A\right)<1$, and matrix $J$ share the same eigenvalues
with $A$, $\left|\lambda_{i}\right|<1$. This lead to $\lim_{k\to\infty}c_{m,n}\left(i,k\right)=0,\:\mbox{for all }1\leq m\leq r_{i},1\leq n\leq r_{i}$,
so that the each Jordan block is convergent. Therefore, the matrix
iteration $A^{k}=TJ^{k}T^{-1}$ is also convergent. This completes
the proof. 
\end{proof}
We give the proof of \prettyref{thm:Convergent <=00003D> p(A)<1}
here as it will be very useful in the proof of an convergence conditions
theorem for distributive consensus algorithm, see \prettyref{sub:DT-First-Order-DAC}.
At the same time, the Jordan normal form weight matrix $W^{k}=TJ^{k}T^{-1}$
gives the local value vector $\mathbf{x}\left(k\right)=W^{k}\mathbf{x}\left(0\right)$
another expression in terms of eigenvalues and eigenvectors, which
reflects the basic ideas of the finite-time consensus algorithm. This
will be introduced in \prettyref{sec:Finite-Time-Distributed-Consensu}. 




\paragraph{Gerschgorin's theorem}

The calculation of eigenvalues a matrix $A$ involves determination
of the matrix $\lambda I-A$ and solving a high order polynomial equation.
In some situations, for example, when the matrix dimension is very
large, it is very difficult to determine the spectral radius precisely.
However, the following theorem of Gerschgorin provides an upper bound
of the spectral radius \cite{Horn1990}.
\begin{thm}
Let $A=\left(a_{i,j}\right)$ be an arbitrary $n\times n$ matrix.
Denote the 
\begin{equation}
d_{i}=\sum_{j=1,j\neq i}^{n}\left|a_{i,j}\right|
\end{equation}
then all the eigenvalues of matrix $A$ are lie in the union of the
disks.
\begin{equation}
\left|z-a_{i,i}\right|\leq d_{i},\ 1\leq i\leq n.
\end{equation}

\end{thm}
The above theorem is well-known, so the proof is omit here. But the
theorem immediately give the result of
\begin{cor}
\label{cor:Upper Bound of p(A)}Let $A=\left(a_{i,j}\right)$ be an
arbitrary $n\times n$ matrix, and 
\begin{equation}
v=\max_{1\leq i\leq n}\sum_{j=1}^{n}\left|a_{i,j}\right|
\end{equation}
then we have $\rho\left(A\right)\leq v$. 
\end{cor}
Therefore, the maximum of the row sums of the modular of the entries
of matrix $A$ provides a upper bound for spectral radius $\rho\left(A\right)$.
Since $A$ and $A^{T}$ have the same eigenvalues, applying the \ref{cor:Upper Bound of p(A)}
to $A^{T}$ will lead to 
\begin{cor}
Let $A=\left(a_{i,j}\right)$ be an arbitrary $n\times n$ matrix,
and 
\begin{equation}
v'=\max_{1\leq j\leq n}\sum_{i=1}^{n}\left|a_{i,j}\right|
\end{equation}
then we have $\rho\left(A\right)\leq v'$.
\end{cor}
As both the row sum and column sum of the modular of the entries of
matrix can provide the upper bound for $\rho\left(A\right)$, the
minimum of $v$ and $v'$ gives the better upper bound. 



Just as we assume before, the adjacent matrix ${\cal A}$ of the graph
${\cal G}$ is non-negative, which means all elements are non-negative.
Then, the induced graph Laplacian matrix will have the following result
using the Gerschgorin's theorem. 


\paragraph{Generalize from Gerschgorin's theorem}
\begin{thm}
\cite{Olfati-Saber2004} \label{thm:Bounds of eig(L)}Let the graph
has the Laplacian matrix ${\cal L},$ denote the maximum node outdegree
of the graph by 
\begin{equation}
d_{max}=\max_{1\leq i\leq n}\left(\sum_{j=1,j\neq i}^{n}l_{i,j}\right)
\end{equation}
Then, all the eigenvalues of ${\cal L}$ are located in the following
disk,
\begin{equation}
\left|z-d_{max}\right|\leq d_{max}
\end{equation}
which is centered at $z=d_{max}+0j$ on the complex plane.\end{thm}
\begin{proof}
Based on the Gerschgorin's theorem, all the eigenvalue of ${\cal L}$
are located in the union of the disks.
\begin{equation}
\left|z-l_{i,i}\right|\leq\sum_{j=1,j\neq i}^{n}\left|l_{i,j}\right|,\ 1\leq i\leq n.\label{eq:disk Union of eig(L)}
\end{equation}
Since ${\cal A}=\left[a_{i,j}\right]$ is an non-negative matrix,
by the definition of Laplacian matrix, $l_{i,j}\leq0$ and $l_{i,i}=\sum_{k=1,k\neq i}^{n}a_{ik}\geq0.$
Therefore, let $d_{i}=l_{i,i}$ and 
\begin{equation}
d_{i}=\sum_{j=1,j\neq i}^{n}\left|l_{i,j}\right|
\end{equation}
 and the union of disks becomes
\begin{equation}
\bigcup_{1\leq i\leq n}\left\{ z\in\mathbb{C}:\left|z-d_{i}\right|\leq d_{i}\right\} 
\end{equation}
On the other hand, all these disks are located inside the largest
disk with radius $d_{max}$. This result ends the proof of the theorem
\end{proof}
Based on the \prettyref{thm:Bounds of eig(L)}, it is obvious all
the nonzero eigenvalues of ${\cal L}$ have positive real parts. This
immediately lead to a convergence theorem of the continuous-time consensus
protocol \prettyref{eq:1st Order Protocol}. The solution given in
exponential matrices form \prettyref{eq:x(t) of CT-1st-DAC} converge
to a consensus value as $t\to\infty$. Since all the nonzero eigenvalue
of $-{\cal L}$ located in the disk $\left|z+d_{max}\right|\leq d_{max}$,
and the eigenspace associated with zero is one-dimensional. Consequently,
the eigenvector associated with zero eigenvalue has the form $\alpha\mathbf{1}$,
i.e. $x_{i}=\alpha$ for all $i$. This result will be very helpful
as the negative real part can guarantee that the system dynamic is
stable and convergent. We will show the proof after introduce the
well-known Perron-Frobenius Theorem. 


\paragraph{Perron-Frobenius Theorem}

The Perron-Frobenius theorem states that if matrix $A$ is nonnegative
and irreducible which means the digraph of matrix A is strongly connected,
the spectral radius $\rho(A)$ is equal to a simple eigenvalue of
$A$ associated with a positive eigenvector \cite{Piziak2007}. The
details of Perron-Frobenius is as follows.
\begin{thm}
\label{thm:Perron-Frobenius thm.} Let $A$ be an $n\times n$ and
irreducible matrix with non-negative and real numbers as its entries.
Then, \end{thm}
\begin{enumerate}
\item $A$ has a positive real eigenvalue equal to its spectral radius $\rho\left(A\right)$.
\item $\rho\left(A\right)$ is a simple eigenvalue of $A$.
\item To the eigenvalue $\rho\left(A\right)$, the corresponding eigenvector
is positive, i.e. $\mathbf{x}>\mathbf{0}$.
\item $\rho\left(A\right)$ increases if any entry of $A$ increases.
\end{enumerate}
Recall the problem of finding the bounds for the spectral radius,
the Perron-Frobenius theorem provides the nontrivial lower-bound of
$\rho\left(A\right)$. In the \prettyref{cor:Upper Bound of p(A)},
the nontrivial upper bound of $\rho\left(A\right)$ is found. Together
with these two results, we can have a conclusion of the spectral radius
of a non-negative and irreducible matrix given in the following.
\begin{lem}
If $A=\left[a_{i,j}\right]$ is an $n\times n$ non-negative and irreducible
matrix, then either 
\begin{equation}
\sum_{j=1}^{n}a_{i,j}=\rho\left(A\right),\mbox{for all }1\leq i\leq n,
\end{equation}
or 
\begin{equation}
\min_{1\leq i\leq n}\left(\sum_{j=1}^{n}a_{i,j}\right)<\rho\left(A\right)<\max_{1\leq i\leq n}\left(\sum_{j=1}^{n}a_{i,j}\right).
\end{equation}
\end{lem}
\begin{thm}
Let $A=\left[a_{i,j}\right]$ be an $n\times n$ non-negative and
irreducible matrix, for any $\mathbf{x}>\mathbf{0}$, either
\begin{equation}
\min_{1\leq i\leq n}\left(\frac{\sum_{j=1}^{n}a_{i,j}x_{j}}{x_{i}}\right)<\rho\left(A\right)<\max_{1\leq i\leq n}\left(\frac{\sum_{j=1}^{n}a_{i,j}x_{j}}{x_{i}}\right)
\end{equation}
or 
\begin{equation}
\frac{\sum_{j=1}^{n}a_{i,j}x_{j}}{x_{i}}=\rho\left(A\right),\ \mbox{for all }1\leq i\leq n.
\end{equation}
Moreover,
\begin{equation}
\max_{\mathbf{x}\in P}\min_{1\leq i\leq n}\left(\frac{\sum_{j=1}^{n}a_{i,j}x_{j}}{x_{i}}\right)=\rho\left(A\right)=\min_{\mathbf{x}\in P}\max_{1\leq i\leq n}\left(\frac{\sum_{j=1}^{n}a_{i,j}x_{j}}{x_{i}}\right)
\end{equation}

\end{thm}
The equality is valid if we choose the $\mathbf{x}$ equal to the
positive eigenvector $e>0$ corresponding to the eigenvalue $\rho\left(A\right)$.
The method shown above will be applicable, because it provides us
both the upper bounds and lower bounds for the spectral radius of
a non-negative and irreducible matrix, by a simple algorithm without
calculating the determination of $\lambda I-A$. 


\paragraph{Generalized from Perron-Frobenius theorem}

The result induced from Perron-Frobenius theorem gives the upper bound
and lower bound of the matrix. Considering the system dynamic $x\left(t\right)=\exp\left(-{\cal L}t\right)x\left(0\right)$.
Because $\exp\left(-{\cal L}t\right)$ is a non-negative matrix, the
Perron-Frobenius theorem tell us that $\exp\left(-{\cal L}t\right)$
has a positive real eigenvalue equals to one which is also the spectral
radius. Together with the \prettyref{thm:Bounds of eig(L)} which
implies that all the eigenvalues of $-{\cal L}$ have negative real
part, we immediately come to the following theorem.
\begin{thm}
Assume ${\cal G}=\left({\cal V},{\cal E},{\cal A}\right)$ is strongly
connected graph, and the associated graph Laplacian matrix ${\cal L}$
is defined in \prettyref{eq:Graph Laplacian def.}, which has only
one zero eigenvalue. Let the $u_{r}$ is the uniformed right eigenvector
and $u_{l}$ is the uniformed left eigenvector associated with the
zero eigenvalue, i.e. $Lu_{r}=0$, $u_{l}^{T}L=0$, then we have $u_{l}^{T}u_{r}=1$
and the system dynamic 
\begin{equation}
x\left(t\right)=\exp\left(-{\cal L}t\right)x\left(0\right)
\end{equation}
will have the stable state of the system dynamic given by

\begin{equation}
x\left(t\right)=Kx\left(0\right)
\end{equation}
 where K is a matrix in $R^{n}$, and $K=\lim_{t\to\infty}\exp\left(-{\cal L}t\right)=u_{r}u_{l}^{T}$. \end{thm}
\begin{proof}
Let $A=-{\cal L}$, and it has a Jordan form of $A=UJU^{-1}$. Then
we can have $\exp\left(At\right)=U\exp\left(Jt\right)U^{-1}$. Because
$A$ has all its eigenvalue except a simple zero eigenvalue have negative
real part, thus, as $t\to\infty$, all other Jordan block vanish,
and $\exp\left(Jt\right)$ converges to a matrix with only single
nonzero entry, denoted by $Q$. Since matrix $U$ contains a column
which associated with the zero eigenvalue of $A$ is $u_{r}$, similarly,
$U^{-1}$ has a corresponding row equals to $u_{l}$. By simply calculating
the equation $K=UQU^{-1}$, we can show that $K=u_{r}u_{l}^{T}$.
And the fact $U^{-1}U=I$ shows that $u_{l}^{T}u_{r}=1$ . This ends
the proof. 
\end{proof}
This is an important theory for continuous-time consensus problem.
It provides the necessary and sufficient conditions of the graph Laplacian
matrix so that a convergent consensus algorithm could be carried out
on the network. For the average consensus problem, it is obvious that
all the elements in $K$ must equal to $\frac{1}{n}$. This requires
the graph Laplacian ${\cal L}$ satisfies the conditions: ${\cal L}\mathbf{1}=0$,
$\mathbf{1}^{T}{\cal L}=0$, where $\mathbf{1}\in R^{n}$ is an all
unity vector. And the $u_{r}$ and $u_{l}$ will change into vectors
with equivalent constant in all its components. If they are uniformed,
then $u_{r}=u_{l}=\frac{1}{\sqrt{n}}.$ (todo Give a figure about
the eigenvalue of Laplacian matrix)


\subsubsection{Diagonalizable matrix \& Symmetric matrix}

The Jordan normal form of weight matrix $W^{k}=TJ^{k}T^{-1}$
gives the local value vector $\mathbf{x}\left(k\right)=W^{k}\mathbf{x}\left(0\right)$
an analitical expression in terms of eigenvalues and eigenvectors.
Moreover, if the matrix $W$ is symmetric, the expressions
of $\mathbf{x}\left(k\right)$ can be simplified. Then, algorithms
such as the finite-time consensus, can be implemented more easily.
In addition, under the the assumption of symmetric weight matrices,
the optimal weight matrix which has the fastest convergence rate can
be found through a semi-definite problem. \cite{Li2010}. 
\begin{defn}
A matrix $A$ is diagonalized, if and only if there exist a nonsigular
matrix $T$, which reduce $A$ into the form 
\[
T^{-1}AT=\mbox{diag}\left(\lambda_{1},\lambda_{2},\ldots,\lambda_{n}\right)
\]

\end{defn}
Generally, a matrix are diagonalized by an unitary matrix if and only
if it is normal. Normal matrix $A$ means it satisfies $A^{H}A=AA^{H}$.
Let $\lambda_{1},\lambda_{2},\ldots,\lambda_{n}$ be the eigenvalues
of $A$. There exists a unitary matrix $U$, which satisfies $U^{H}=U^{-1}$,
such that $U^{-1}AU=\mbox{diag}\left(\lambda_{1},\lambda_{2},\ldots,\lambda_{n}\right)$.
For real normal matrix $A$, if all of its eigenvalues are real, there
exists an orthogonal matrix $P,$ which has $P^{T}=P^{-1}$, reduces
the real matrix $A$ into diagonalized form $P^{-1}AP=\mbox{diag}\left(\lambda_{1},\lambda_{2},\ldots,\lambda_{n}\right)$. 

Specificlly, any real nonsigular symmetric matrix $A\in R^{n\times n}$
has $n$ linearly independent real eigenvectors. Moreover, these eigenvectors
can be chosen so that they are orthogonal to each other with modular
one. Thus, the real symmetric matrix can be decomposed by an orthogonal
matrix $P$, i.e. $A=P\Lambda P^{-1}$, where $\Lambda$ is a diagonal
matrix whose entries equal to the eigenvalues of $A$. Also the symmetric
matrix has all its eigenvalues algebra multiplicity equal to the gerometric
multiplicity, so all the Jordan block have size one.
