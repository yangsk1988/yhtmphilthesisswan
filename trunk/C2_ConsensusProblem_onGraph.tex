
\section{\label{sec:Consensus-problem-on}Consensus problem on Graphs (todo)}

Consider a network (connected graph) $\mathcal{G}=\left(\mathcal{V},\mathcal{E},{\cal A}\right)$
be a weighted digraph (or undirected graph) consisted by a set of
nodes $\mathcal{V}=\left\{ 1,2,...,n\right\} $, a set of edges $\mathcal{E}\subseteq\mathcal{V}\times\mathcal{V}$
and a weighted adjacency matrix ${\cal A}=\left[a_{ij}\right]$. The
edge $\left(i,j\right)\in\mathfrak{\mathcal{E}}$ is an unordered
pair of distinct nodes, associated with an element in the adjacency
matrix ${\cal A}$, i.e. $\left(i,j\right)\in\mathfrak{\mathcal{E}}\Leftrightarrow a_{ij}$.
we assume $a_{ij}=0$, if $j\notin{\cal N}_{i}$ (note that $a_{ii}=0$,
as $i\notin{\cal N}_{i}$) for all $i\in\mathcal{V}$.. Moreover,
the set of neighbors of node $i$ is denoted by $\mathcal{N}_{i}=\text{ }\left\{ j\in{\cal V}|\left(i,j\right)\in\mathcal{E}\right\} $,
an node $i$ can only transmit information to other nodes that belong
to the neighbors set. (OK1)

Suppose each node holds an initial scalar value $x_{i}\left(0\right)\in R$,
which can be a value that locally acquired by node representing physical
quantities including temperature, humidity, illumination, attitude,
and so on. We define the local value vector $\mathbf{x}\left(0\right)=\left[x_{1}\left(0\right),...,x_{n}\left(0\right)\right]^{T}\in R^{n}$
to represent all the initial values on the network. The network is
said to be reached a consensus if and only if $x_{i}=x_{j}$, for
all nodes $i,j\in{\cal V},i\neq j$. In the other words, all the nodes
of the network are in an agreement, the common value of all nodes
is called the consensus value. (OK1)

To describe the behavior of each node or agent, suppose each node
has the following dynamics 

\begin{equation}
\dot{x}_{i}=f\left(x_{i},u_{i}\right),i\in{\cal V}
\end{equation}
and the graph (or network) is a system has the dynamics

\begin{equation}
\mathbf{\dot{x}}=F\left(\mathbf{x},\mathbf{u}\right)\label{eq:system dynamic}
\end{equation}
where $F\left(\mathbf{x},\mathbf{u}\right)$ is the columnwise concatenation
of individual dynamics $f\left(x_{i},u_{i}\right)$, for all node
$i=1,\ldots,n$. In an ad-hoc network with the mobile nodes, the topology
$G$\textbf{ }is switching and the system will updating its $F\left(\mathbf{x},\mathbf{u}\right)$
from time to time. 

The input or feedback $u_{i}$ in the node's dynamic is a function
of the historical states of node $i$ and its neighbors
\begin{equation}
u_{i}=g\left(x_{j_{1}},x_{j_{2}},\ldots,x_{j_{m_{i}}}\right)\label{eq:consensus protocol}
\end{equation}
where $j_{1},\ldots,j_{m_{i}}$ are the node indexes that belong to
the set $\left\{ i\right\} \cup{\cal N}_{i}$. \prettyref{eq:consensus protocol}
is called a consensus protocol under topology $G$. If the network
graph is not fully connected, it is said to be a distributive consensus
protocol. 
\begin{defn}
Let ${\cal X}:R^{n}\to R$ be a function of $n$ variables of $x_{1},x_{2},\ldots,x_{n}$
and let $\mathbf{x}\left(0\right)$ denotes the initial condition
of the network. The ${\cal X}$-consensus problem is a distributive
method to calculate the consensus value ${\cal X}\left(\mathbf{x}\left(0\right)\right)$
in a graph $G$. 
\end{defn}
Actually, there are different consensus problems too. For instance,
the average consensus ${\cal X}\left(\mathbf{x}\right)=\frac{1}{n}\sum_{i=1}^{n}x_{i}\left(0\right)$,
maximum consensus ${\cal X}\left(\mathbf{x}\right)=\max\left(\mathbf{x}\right)$,
minimum consensus ${\cal X}\left(\mathbf{x}\right)=\min\left(\mathbf{x}\right)$
and variance consensus ${\cal X}\left(\mathbf{x}\right)=\mbox{var}\left(\mathbf{x}\right)$
are given by their expressions respectively. The average consensus
is a special case of consensus problem, which computing the average
of all initial values $\overline{x}=\mbox{mean}\left(\mathbf{x}\right)=\frac{1}{n}\sum_{i=1}^{n}x_{i}\left(0\right)$
using a distributive system dynamics $\mathbf{\dot{x}}=F\left(\mathbf{x},\mathbf{u}\right)$
in a network $G$.(OK1)

We are interested in the distributive solutions of the consensus problem
as the network only allows an node to communicate with its neighbors.
We say the protocol \prettyref{eq:consensus protocol} solves the
consensus problem asymptotically if and only if there exists an asymptotically
stable state $x^{*}={\cal X}\left(\mathbf{x}\right)$ of system dynamics
\prettyref{eq:system dynamic}, which satisfying for all $\delta>0$,
there exists a time index $t^{*}>0$, such that $\left|x_{i}(t)-x^{*}\right|<\delta$
for all $t>t^{*}$ and $\forall i\in{\cal V}$.

The average consensus problem is much challenge than the maximum consensus
(or minimum consensus), since it relates a linear combination of all
the initial states of network nodes and the condition $x_{i}^{*}={\cal X}\left(\mathbf{x}\right)$
for all nodes $i$ has to be satisfied. Furthermore, the variance
consensus problem can be solved by two instances of average consensus,
because we have the relation $\mbox{var}\left(\mathbf{x}\right)=\mbox{mean}\left(\mathbf{x}\cdot\mathbf{x}\right)-\left[\mbox{mean}\left(\mathbf{x}\right)\right]^{2}$.
Thus, in the following of this thesis we focus on the average consensus
problem and the distributed average consensus (DAC) algorithms. (OK1)


\subsection{Continuous-time vs. Discrete-time Consensus}

In this section, we illustrate consensus protocol 
\begin{equation}
u_{i}\left(t\right)=\sum_{j\in{\cal N}_{i}}a_{ij}\left(x_{j}-x_{i}\right)\label{eq:1st Order Protocol}
\end{equation}
that solves the average consensus problem ${\cal X}\left(\mathbf{x}\right)=\mbox{mean}\left(\mathbf{x}\right)$
and show the difference and similarity of continuous-time with the
dynamics 

\begin{equation}
\dot{x}_{i}\left(t\right)=u_{i}\left(t\right)\label{eq:continuous-time consensus}
\end{equation}
and discrete-time consensus with the dynamics 
\begin{equation}
x_{i}\left(k+1\right)=x_{i}\left(k\right)+u_{i}\left(k\right)\label{eq:discre-time consensus}
\end{equation}


The continuous-time consensus requires the nodes in a network have
dynamics in a form of differential equations while nodes in the discrete-time
consensus has dynamics in difference equation. Continuous-time consensus
involves analog signals that are easily interfered by channel noise.
Consequently, the finally consensus value will be a random variable
with mean equals to the consensus value without noise and variance
proportional to the signal noise ratio. It is shown in \cite{Kar2009},
the variance is also proportional to time $t$ hence it is increasing
as the algorithm executing. 

(todo give a simulation with channel noise)

Without the channel noise, the system dynamics with the feedback in
\ref{eq:1st Order Protocol} can evolve into a linear system with
differential equation given by 
\begin{equation}
\mathbf{\dot{x}}\left(t\right)=-{\cal L}\mathbf{x}\left(t\right)\label{eq:system differential dynamics}
\end{equation}
Solving the differential equation \ref{eq:system differential dynamics}
will yield a continuous-time solution in an exponential matrices form
\begin{equation}
\mathbf{x}\left(t\right)=\exp\left(-{\cal L}t\right)\mathbf{x}\left(0\right)\label{eq:x(t) of CT-1st-DAC}
\end{equation}
where ${\cal L}$ is called the weighted graph Laplacian associated
with network graph ${\cal G}$, which is defined by

\begin{equation}
l_{ij}=\begin{cases}
\sum_{k=1,k\neq i}^{n}a_{ik} & j=i\\
-a_{ij} & j\neq i
\end{cases}\label{eq:Graph Laplacian def.}
\end{equation}
For a graph with 0-1 adjacency, the graph Laplacian can be denoted
in another form, which is unweighted Laplacian matrix denoted by $\mathbf{L}$

\begin{equation}
l_{ij}=\begin{cases}
\left|{\cal N}_{i}\right| & j=i\\
-1 & j\in{\cal N}_{i}\\
0 & \mbox{otherwise}
\end{cases}\label{eq:Laplacian def.}
\end{equation}
In some literature for example \cite{Xiong2009a}, they use the second
definition (\prettyref{eq:Laplacian def.}) of the Laplacian matrix
to analyze the convergence rate of DAC algorithm. It is a special
case when weights $a_{ij}$ for all edges in ${\cal E}$ equal to
one. Therefore, to distinguish them, we denote the weighted graph
Laplacian matrix and the unweighted Laplacian matrix by ${\cal L}$
and $\mathbf{L}$ respectively. 

On the other hand, discrete-time consensus only involves the quantization
error during the algorithm execution as long as the data packages
are correctly received. Nowadays, sensor nodes with digital processing
unit becomes more cheaper, we can get more benefits from the advantages
of discrete-time consensus algorithms. Consequently, discrete-time
consensus algorithms are mainly discussed in the following of this
thesis. 

For network agents have discrete-time consensus protocol, their system
dynamics can be given in a matrix form
\begin{equation}
\mathbf{x}(k+1)=\mathbf{W}\mathbf{x}(k)
\end{equation}
where $\mathbf{W}=\left[w_{ij}\right]=I-{\cal L}$. We say the iteration
is convergent if there exists a vector denoted by $\mathbf{x}^{*}$,
which satisfies.

\begin{equation}
\mathbf{x}^{*}=\mathbf{W}\mathbf{x}^{*}\label{eq:stable state iteration}
\end{equation}
Moreover, 

\begin{equation}
\mathbf{x}(k+1)-\mathbf{x}^{*}=\mathbf{W}\left[\mathbf{x}(k)-\mathbf{x}^{*}\right]\label{eq:error vector iter. 1st}
\end{equation}
\prettyref{eq:stable state iteration} states that $\mathbf{x}^{*}$
is a right eigenvector of matrix $\mathbf{W}$ associated to a simple
eigenvalue 1. For more details about the discrete-time first-order
DAC algorithm, see \prettyref{sub:DT-First-Order-DAC}. 
