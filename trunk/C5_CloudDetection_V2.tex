
\section{Distributed Detection of Cloud}

A cloud is a group of liquid or solid particles floated in the air.
Sometime it contains harmful or dangerous particles so that it needs
to be observed and tracked. Laser technology enables the remote detection
of cloud. In real environment, the received signals may be interfered
by noise or corrupt by moving object. Thus, a method to use distributed
detection is necessary. Gaussian plume model is widely used in cloud
plume modeling, which could be used to describe the mean value of
cloud concentration. However, the sensor's observation is actually
a random process with mean and variance values. So an expectation-maximization
algorithm is adopted to get Gaussian mixture model that gives the
distribution of the sensor\textquoteright{}s observation. Then, the
decision of cloud existence could be made based on the log likelihood
ratio.


\subsection{Introduction }

(todo add reference)

A cloud is a group of liquid or solid particles floated in the air.
Sometime it contains harmful or dangerous particles so that it needs
to be observed and tracked. There are various types of sensors can
perform the detection. According to their types of targets, the detection
can be categorized into smoke/gas/aerosol detection. On the other
hand, based on their sensing methods, there are also divided into
contact detection and remote sensing. (OK1)

Smoke and gas detection is very common in daily life. For example,
the fire alarm sensor in building, which is contact sensing device.
When the sensing element contacts with smoke or gas agent, the chemical
or physical reactions change the electrical characteristics of the
sensing element. Then the alarm is set if the agent's concentration
is larger than a threshold. (OK2)

Smoke detection based on video and image processing is also possible.
For example, in the wildfire video surveillance, people try to change
the human based surveillance into automatic smoke/fire detection.
These intelligent algorithms extract smoke's features based on video
signal, and then classify the objects in the video as smoke or non-smoke.
This also attracts much research interest. (OK2)

Some time, the agent does not directly contact with the sensing element.
The device pumps in some air and illuminate the air with multiple
wavelengths to get the absorption spectrum, which normally obtained
by an infrared spectroscopy. Then the concentration or species of
the agent can be estimated when prior knowledge is available. (OK1)

When sensors are not able to contact with cloud in the sky, laser
technology enables the remote detection. The concept of remote sensing
is very close to that of a spectroscopy. It is an optical technology
that can measure the distance to a target, or other properties of
a target by illuminating the target with light, often using pulses
from a laser. The devices typically used for studies of aerosols and
clouds remotely is called LIDAR. (OK1)

When the cloud is illuminated by a laser beam, the particles absorb
the energy and emit fluorescent light, as well as ``reflect'' light
back to the source (referred as backscatter). The back-scattered light
wavelength is identical to the transmitted light \cite{Lidar.Wiki.2011},
and the magnitude of the back-scattered light at the given range depends
on the back-scatter coefficient of scatterers and the extinction coefficients
of the scatterers along the path at that range \cite{P.M.Hamilton1969}.
The ``fingerprint'' of the fluorescent light or backscatter can
be an evidence of the concentration or species of particles in that
cloud. The microprocessor in the sensor node could identify the received
signals, and make a declaration of cloud. (OK1)

In battle field applications, aerosol detection may also related to
bio-aerosol detection. As the bio-aerosol release by a biological
weapon is extremely dangerous to any biological element in that area.
It is necessary to detect and discrimination it as soon as possible.
The detection and discrimination can be done by a remote sensing lidar,
which illuminate the bio-aerosol  and collect back-scatters by a telescope
just similar to the spectroscopy. Therefore, the agent could be discriminated
according to its spectrum. In research simulation, the bio-aerosol
is often created by spread bacillus subtilis spores in the air. However,
the reflection and extinction coefficients are closely related to
the particles species and absorption spectrum are different from one
agent to another. To estimate the concentration of an aerosol based
on backscatter spectrum requires prior knowledge of the wavelength-dependent
backscatter coefficients.(OK1)

There are many challenges in outdoor environment for this method.
First, the background radiation (for example, sunshine in the day
time) will ruin the lidar signals. But it can be compensated by increase
the laser power, add optical filter in front of the telescope or tracking
the background radiation level. Second, the interference of moving
objects in the sky like birds and balloons will have a very high reflect
coefficient compared with gases. Third, failure of sensor nodes. Because
of the properties of sensor nodes, such as energy constrains, vulnerable
to intruders, they often malfunction and become unreliable. Fourth,
Cloud is a diffusive target. The diffusion target will have different
concentration from one sensor to another, so the distribution of cloud
has random variables that capture the special variation of concentration
\cite{N.Kh.2004}. Finally, a lidar with discrimination ability is
very expensive so that it often performs passive detection and discrimination.
Active detection of the bio-aerosol is taken by some other sensors
distributing in the battle field. So this comes to our problem: detecting
the bio-aerosol with sensors distributed in the environment with noise
and interference. (OK1)

To overcome unreliability of sensor nodes and interference for outdoor
remote cloud detection, the distributed detection method could be
adopted.

Signal can be processed distributively by distributed consensus algorithm
(DCA). The word consensus means each node would have an agreement
on the declaration of the target after the algorithm. In addition,
each node only broadcast its local value until the algorithm converges
\cite{Xiao2004}. This method can save much energy for nodes that
heavy loaded in the data gathering. (OK2)


\subsection{Background and System model }

This section is structured in the following way, section \ref{sub:Cloud-Detection-Scenario}
shows the system model. Section \ref{sub:Received-Signal'-Model}
illustrates the sensor observation model and explain some techniques
in distributed detection of cloud. In the training stage, expectation-maximization
algorithm is used to get Gaussian mixture model for sensor observation
from background noise and target. In the following, it is the detection
stage, where each sensor calculate its local log likelihood ratio,
and substitute it into distributed consensus algorithm. When the algorithm
converges, each sensor in the network can have an agreement on declaring
the cloud or not. Finally, simulation and result is give in section
\ref{sec:Simulation}. To get more real cloud data, 3D fluid animation
techniques with turbulence flow were used. The performance of this
detection system is also given. It shows the multiple sensor detection
will be more reliable in the noisy environment.


\subsubsection{Cloud Detection Scenario\label{sub:Cloud-Detection-Scenario}}

In the cloud detection scenario shown by Figure \ref{fig:Cloud-detection-scenario},
a source produces cloud in a fixed position with fixed power; a time
invariant wind velocity parallel to the ground, blows the cloud to
the positive direction of the x-axis. With these parameters available,
the Gaussian plume model \cite{Lin1996}\cite{GPM.JA.2011} is used
to describe the cloud concentrate at any position $\left(x,y,z\right)$.
On the ground, multiple sensors (1 to $L$) aim to the plume perpendicularly
to the ground plane and don't change their positions. The positions
were chosen so that the laser beams can penetrate into the plume and
backscatters are observed by the laser receivers. 

\begin{figure}
\hfill{}\subfloat[\label{fig:Cloud-detection-scenario}Cloud detection scenario]{\hfill{}\includegraphics[bb=0bp 0bp 342bp 225bp,clip,width=7cm]{\string"D:/Dropbox/PaperWork/CloudDetection/Cloud Detection Task Description/SystemModel/SysModel1\string".pdf}\hfill{}

}\hfill{}\subfloat[\label{cap:GaussianPlume}Gaussian plume model.]{\hfill{}\includegraphics[bb=20bp 10bp 1840bp 1210bp,clip,width=7cm]{D:/Dropbox/PaperWork/CloudDetection/GaussianPlumeImage/A9RF2DF\lyxdot tmp}\hfill{}

}\hfill{}

\caption{System model}
\end{figure}





\subsubsection{Diffusive Cloud Model}

Since 1970's, The Gaussian plume model is widely used in cloud plume
concentration modeling, which can describe the mean value of the cloud
concentration at any position \cite{Shieh1972}. However, The cloud
concentration is actually a random process in a real plume, the Gaussian
plume model can not describe its variance. If we observing the real
cloud plume for a long time and take the average, the result is Gaussian
plume model. (OK1)

With the computer graphic technology of 3D fluid simulation \cite{He2011},
a 3D cloud animation is implemented to get more real cloud plume while
fluid dynamics and wind turbulence is considered. (OK1)

This section first introduce the Gaussian plume model mentioned in
some state-of-art applications. Then, some expressions are driven
based on the diffusive equations to obtain the modified Gaussian plume
model especially for laser detection. Finally, cloud targets are generated
by 3D cloud animation. A bunch of cloud plumes is simulated to generate
enough data of cloud concentration for algorithm testing. (OK1)


\paragraph{Gaussian Plume Model }

\cite{N.Kh.2004} gives a model of the concentration values $C$ of
pollutants to be emitted by point instantaneous source at height $H$,
described by the normal (Gaussian) distribution 

\begin{eqnarray}
C(x,y,z,t) & = & \frac{Q}{(2\pi)^{3/2}\sigma_{x}\sigma_{y}\sigma_{z}}\exp\left(\tfrac{-\left(x-ut\right)^{2}}{2\sigma_{x}^{2}}\right)\exp\left(\tfrac{-\left(y-vt\right)^{2}}{2\sigma_{y}^{2}}\right)...\label{eq:Point source concentration}\\
 &  & \left(\exp\left(\tfrac{-\left(z-H-wt\right)^{2}}{2\sigma_{z}^{2}}\right)+\exp\left(\tfrac{-\left(z+H-wt\right)^{2}}{2\sigma_{z}^{2}}\right)\right)
\end{eqnarray}
where $t$ is the time, $Q$ the source emission power, $u,v,w$ are
the orthogonal components of wind velocity, $\sigma_{x},\sigma_{y},\sigma_{z}$
are the horizontal and vertical dispersions, $H$ the source height
. 

To describe the continuous cloud source emitted into the air. An integration
form $t=0$ to $\infty$ is taken for Eq.\ref{eq:Point source concentration}.
So for continuous source the model is not related to the time $t$,
 The model after integration is called Gaussian plume model. To make
things easy, assume the wind velocity components $v=0,w=0$. The Gaussian
plume model changes into

\begin{eqnarray}
C(x,y,z,H) & = & \int_{0}^{\infty}C(x,y,z,t)dt\label{eq:Continous source C}\\
 & = & \frac{Q}{2\pi u\sigma_{y}\sigma_{z}}\exp\left(\tfrac{-y^{2}}{2\sigma_{y}^{2}}\right)\left(\exp\left(\tfrac{-\left(z-H\right)^{2}}{2\sigma_{z}^{2}}\right)+\exp\left(\tfrac{-\left(z+H\right)^{2}}{2\sigma_{z}^{2}}\right)\right)\nonumber 
\end{eqnarray}
where$\sigma_{y}=ax^{b}$, $\sigma_{z}=cx^{d}$, $a,b,c,d$ are the
coefficients which relate to the atmospheric stability are shown in
Table \ref{tab:The-parameter-of}.



\begin{table}
\caption{\label{tab:The-parameter-of}The parameter of a,b,c,d according to
atmospheric stability}
\hfill{}%
\begin{tabular}{|c|c|c|c|c|}
\hline 
Atmospheric stability  & a & b & c & d\tabularnewline
\hline 
\hline 
A & 0.527 & 0.865 & 0.28 & 0.90\tabularnewline
\hline 
B & 0.371 & 0.866 & 0.23 & 0.85\tabularnewline
\hline 
C & 0.209 & 0.897 & 0.22 & 0.80\tabularnewline
\hline 
D & 0.128 & 0.905 & 0.20 & 0.76\tabularnewline
\hline 
E & 0.098 & 0.902 & 0.15 & 0.73\tabularnewline
\hline 
F & 0.065 & 0.902 & 0.15 & 0.73\tabularnewline
\hline 
G & 0.046 & 0.902 & 0.10 & 0.62\tabularnewline
\hline 
\end{tabular}\hfill{}
\end{table}



\paragraph{Modified Gaussian Plume Model For Laser Detection}

Because laser emitted by optical sensor is penetrating the could,
the received signal at each sensor can be simplified to be proportional
to the integration of concentration along the line of laser. Therefore,
based on the same diffusion equation, the Gaussian plume model needs
some modifications. 

Similar to the heat diffusion model, the cloud's diffusive behavior
can be described by the 3-dimensional partial derivative equation. 

\begin{equation}
\frac{\partial C}{\partial t}=D_{x}\frac{\partial^{2}C}{\partial^{2}x}+D_{y}\frac{\partial^{2}C}{\partial^{2}y}+D_{z}\frac{\partial^{2}C}{\partial^{2}z}\label{eq:3D diffusion Eq.}
\end{equation}
where $C$ is cloud concentration, and $D_{x},D_{y},D_{z}$ are the
diffusion coefficients along three axis respectively. 

This equation indicates that the rate of density change is proportional
to the curvature of cloud concentration. The density increases where
curvature is positive and decreases where it is negative. If the cloud
is released instantaneously at a single point, the spatial distribution
will be a 3-dimensional normal distribution.

In consideration of the isotropic diffusion case, the diffusion equation
can be simplified. which means $D_{x}=D_{y}=D_{z}=D$. Assume a point
instantaneous source, located at the origin starts to release cloud
at time $t=0$, the solution to \ref{eq:3D diffusion Eq.} is 

\begin{equation}
C(x,y,z,t)=\frac{Q}{\left(4\pi Dt\right)^{3/2}}exp\left(-\frac{x{}^{2}+y^{2}+z^{2}}{4Dt}\right)
\end{equation}
Where $Q$ is the power of the point source. This solution can be
verified by taking partial derivative for both sides.

In addition, if the surrounding air is assumed to be moving towards
the positive direction of x-coordinate in a constant velocity $u$
. The model changes into

\begin{equation}
C(x,y,z,t)=\frac{Q}{\left(4\pi Dt\right)^{3/2}}\cdot exp\left(\frac{(x-ut){}^{2}+y^{2}+z^{2}}{4Dt}\right)\label{eq:3D point Model}
\end{equation}


For point and continuous source at origin, an integration form $t=0$
to $T$ is taken to find the concentration distribution. The integration
of Eq.\eqref{eq:3D point Model} is very hard to find without the
help of computer, as in the denominator contains $t^{\frac{3}{2}}$.
However, some later research have found the analytical integration
of the atmospheric diffusion equation \cite{Lin1996}. 

As $T\rightarrow\infty$ , the concentration model for continuous
source evolves into

\begin{eqnarray}
C(x,y,z) & = & \int_{0}^{\infty}C(x,y,z,t)dt
\end{eqnarray}
as the laser penetrate the could, the received signal at each sensor
can be simplified to be proportional to the integration of concentration
along the line of laser. Therefore, the received signal 

\[
S(x,y)=\int_{-\infty}^{\infty}C(x,y,z)dz=\int_{-\infty}^{\infty}\int_{0}^{\infty}C(x,y,z,t)dtdz
\]
since the $\int_{-\infty}^{\infty}\int_{0}^{\infty}\left|C(x,y,z,t)\right|dtdz<\infty$,
the integrations can be swapped. thus, we have 
\[
\int_{-\infty}^{\infty}\int_{0}^{\infty}C(x,y,z,t)dtdz=\int_{0}^{\infty}\int_{-\infty}^{\infty}C(x,y,z,t)dzdt
\]
and 
\[
S(x,y)=\frac{Q}{2\pi D}\cdot exp\left(\frac{xu}{2D}\right)\cdot K_{0}\left(\frac{u\sqrt{x^{2}+y^{2}}}{2D}\right)
\]
where $K_{0}(z)$ is the special case of modified Bessel function
of the second kind $K_{n}(z)$. $K_{0}(z)$ is simplified to 
\[
K_{0}(z)=\int_{0}^{\infty}\mbox{cos}(z\cdot\mbox{sinh}t)dt=\int_{0}^{\infty}\frac{\mbox{cos}(z\cdot t)}{\sqrt{t^{2}+1}}dt
\]
Therefore, the finial model of concentration at point $\left(x,y\right)$
can be given by 
\[
C(x,y)=\frac{Q}{2\pi D}\cdot exp\left(\frac{xu}{2D}\right)\cdot\int_{0}^{\infty}\frac{1}{\sqrt{t^{2}+1}}\mbox{cos}(\frac{u\sqrt{x^{2}+y^{2}}}{2D}\cdot t)dt
\]
The concentration distribution is shown in Figure.\ref{fig:2-D-Gaussian-Plume}.

\begin{figure}
\hfill{}\includegraphics[height=6cm]{\string"D:/Dropbox/PaperWork/CloudDetection/Cloud Detection Task Description/GaussianPlumeModel/Plume_2D\string".pdf}\hfill{}

\caption{\label{fig:2-D-Gaussian-Plume}2-D Gaussian Plume Model (modified
for laser detection)}
\end{figure}



\paragraph{Simulated Cloud by fluid dynamics}

The Gaussian plume model can describe the mean value of the cloud
concentration at any position in the plume, but it can not describe
the cloud concentration variance. With the help of 3D fluid simulation,
a cloud could be more like real while fluid dynamics and wind turbulence
is considered. And this kind of model is simulated to generate a number
of cloud plumes. The data is used in the distributed cloud detection
to training the sensor networks.

As shown in the Figure \ref{fig:Frame-projection}, the frame image
is obtained by taking integration of the 3D raw data over z-dimension.
This is similar to the effect of laser beam penetrating through cloud.
The received light magnitude is the integration of backscatter along
the laser beam. Therefore, pixels value in the frame image is proportional
to the magnitude of sensor observation. In this simulation, they are
treated as equal. After the integration, the pixel values in these
frames is normalized by dividing them with the maximum pixel value
in these frames.

\begin{figure}
\hfill{}\includegraphics[width=14cm]{\string"D:/Dropbox/PaperWork/CloudDetection/Cloud Detection Task Description/Smoke Sequance\string".png}\hfill{}\hfill{}

\caption{\label{fig:Cloud-simulation-frame}Cloud simulation frame sequence,
$192\times256$ pixels for each frame}
\end{figure}


\begin{figure}
\hfill{}\includegraphics[width=8cm]{\string"D:/Dropbox/PaperWork/CloudDetection/Cloud Detection Task Description/SmokeProject/SmokeProjection_PhS\string".pdf}\hfill{}\hfill{}

\caption{\label{fig:Frame-projection}Relationship between 3D cloud concentration
and Sensor's observation. Frame images are obtained by 3D raw data
projection }
\end{figure}





\subsubsection{\label{sub:Received-Signal'-Model}Received Signal Model }

As we know, the cloud concentration variation is caused by turbulence
flow, which is a random process which brings the cloud particles more
far away than the molecular motion. Because of this reason, the cloud
concentration at a given position and the associated sensor's detection
is a random process. 

Suppose the sensor's observation has the following form:

\begin{equation}
x_{l}=\begin{cases}
\mu_{l,0}+n_{l,0} & \mbox{if no cloud exists}\\
\mu_{l,1}+n_{l,1} & \mbox{if cloud exists}
\end{cases}
\end{equation}
where $\mu_{l,m}$ is the mean value of $x_{l}$ depending on hypothesis
$m$; $n_{l,m}$ is the noise of $x_{l}$ depending on hypothesis
$m$ and $n_{l,m}\sim\mathcal{N}\left(0,\sigma_{m}\right)$. Actually,
$\mu_{l,0}$and $\sigma_{l,0}$ denote the mean and variation of background
noise. So let $\mu_{t}$ and $n_{t}$ denote the mean of cloud concentration
and cloud turbulence, we have $\mu_{l,1}=\mu_{l,0}+\mu_{t}$, $n_{l,1}=n_{l,0}+n_{t}$.
These parameters is initialized or calculated in the training stage.

This received signal model doesn't consider interference of moving
objects or be covered by obstacles. To deal with this problems, sensors
may need to build the Gaussian mixture model, which introduced in
\ref{par:Gaussian-mixture-model}. 


\paragraph{\label{par:Gaussian-mixture-model}Gaussian mixture model}

Background radiation and moving objects in the sky have different
strength and variance. When interference exists, the probability of
received signal strength may be a mixture Gaussian distribution. Gaussian
mixture model is a very successful tool in modeling the background
noise in such situation \cite{Stauffer1999}. To build the Gaussian
mixture model, the recent history of a sensor's detection values is
stored. Then, Expectation-Maximization(EM) algorithm \cite{Moon1996}
can be adopted. The model can also be adaptive to track the background
changes. 

Suppose the recent history of a sensor's detections, are given by
$\left\{ x_{1},\ldots,x_{t}\right\} $, which is modeled by a mixture
of $K$ Gaussian distributions. The probability density function of
observing a detection value is

\begin{equation}
f(\mathbf{x}_{t})=\sum_{i=1}^{K}\omega_{i,t}*\eta(\mathbf{x}_{t},\mu_{i,t},\Sigma_{i,t})\label{eq:GMM density function}
\end{equation}
where $K$ is the number of distributions, $\omega_{i,t}$ is the
weight of the $i^{th}$ Gaussian in the mixture at time $t$, $\mu_{i,t}$
is the mean value of the $i^{th}$ Gaussian in the mixture at time
$t$, $\Sigma_{i,t}$ is the covariance matrix of the $i^{th}$ Gaussian
in the mixture at time $t$ , and $\eta$ is a Gaussian probability
density function given by

\begin{equation}
\eta(\mathbf{x}_{t},\mu_{i,t},\Sigma_{i,t})=\frac{1}{\left(2\pi\right)^{\frac{t}{2}}\left|\Sigma\right|^{\frac{1}{2}}}\cdot e^{-\frac{1}{2}\left(\mathbf{x}_{t}-\mu_{t}\right)^{\mathrm{T}}\Sigma^{-1}\left(\mathbf{x}_{t}-\mu_{t}\right)},
\end{equation}
Thus, the distribution of recent detection values are modeled by a
mixture of Gaussian distribution. The recent detections are classified
in $K$ categories. For the background model, all categories correspond
to background noise. When a new detection comes, generally it will
be matched to one of the major components of the model and was used
to update the background model. If a group of adjacent sensors having
detections do not match any categories of background model, it is
more likely the cloud exists. 

Similarly, EM algorithm could also build the Gaussian mixture model
for the observing when cloud exists. In this case, one or more categories
will correspond to the detection mainly raised by clouds reflection.
If a group of adjacent sensors having detections match the categories
of cloud, it is more likely the cloud exists. 

Once the probability density function (PDF) for background and cloud
reflection signals are available, the hypothesis test of cloud existence
is made based on the observation. Normally, this is done by gathering
all the data and taking the log likelihood ratio(LLR) in the fusion
center. In section III-C, distributed consensus algorithm is adopted
to find the LLR, without data gathering or fusion center.


\subsection{\label{sub:Distributed-Cloud-Declaration}Distributed Cloud Detection
Based on DAC algorithm}

The method is organized in two stage, training and detection. The
training stage is to build the probability density function (PDF)
for background radiation and cloud reflection signals (modeled by
a Gaussian mixture model) based on their own observation distributively.

Cloud declaration is normally made through a hypothesis test. When
new observation is acquired, the decision of cloud existence is made
base on the ML or MAP decision rule \cite{Chair1986}, which need
the PDF and all the sensor\textquoteright{}s observation. In a centralized
processing method, these data can be gathered to the fusion center
to perform the data processing. However, they can achieve this distributively
without a fusion center. However, some conditions should be satisfied
to calculating global log likelihood ratio (G-LLR) by distributed
consensus algorithm without the fusion center, and it will be given
later.

Suppose the sensor's observation has the following form:

\begin{equation}
x_{l}=\begin{cases}
\mu_{l,0}+n_{l,0} & \mbox{if no cloud exists}\\
\mu_{l,1}+n_{l,1} & \mbox{if cloud exists}
\end{cases}\label{eq:Cloud signal model}
\end{equation}
where $\mu_{l,m}$ is the mean value of $x_{l}$ depending on hypothesis
$m$ and $n_{l,m}$ is the noise of $x_{l}$ depending on hypothesis
$m$. Actually, $u_{l,0}$and $n_{l,0}$ denote the mean and variation
of background noise. $\mu_{l,1}=\mu_{l,0}+u_{t}$, $n_{l,1}=n_{l,0}+n_{t}$
, $u_{t}$ and $n_{t}$ denote the mean and variation of cloud concentration. 

Suppose the observation of all sensors $x_{1},...,x_{L}$ is available
(for example, gathered by a fusion center) at this moment, we can
have the global log likelihood ratio (G-LLR ) given by

\begin{equation}
LLR(x_{1},...,x_{L})=log\frac{f\left(x_{1},...,x_{L}|H_{1}\right)}{f\left(x_{1},...,x_{L}|H_{0}\right)}\label{eq:G-LLR define Cloud}
\end{equation}
where $f\left(x_{1},...,x_{L}|H_{m}\right)$ is the likelihood function
of $H_{m}$. 

For the received signal model described either by \prettyref{eq:GMM density function}
or \prettyref{eq:Cloud signal model}, if we assume sensor\textquoteright{}s
observation is independent from one to another, the G-LLR can be changed
into the sum of L-LLR. 

\begin{eqnarray}
LLR(x_{1},...,x_{L}) & = & log\frac{f\left(x_{1}|H_{1}\right)\cdot\ldots\cdot f\left(x_{L}|H_{1}\right)}{f\left(x_{1}|H_{0}\right)\cdot\ldots\cdot f\left(x_{L}|H_{0}\right)}=\sum_{i=1}^{L}LLR\left(x_{i}\right)\label{eq:Sum L-LLR Cloud}
\end{eqnarray}
According to Eq.\ref{eq:Sum L-LLR Cloud}, the G-LLR is equal to the
average multiply with the number of sensors in the network. Therefore,
it can be calculated by distributed average consensus (DAC) algorithm.
This is implemented by the following steps: First, each sensor calculates
its local LLR individually; Then, all the sensors update their local
LLR in the DAC iteration until they converge to a common value; Finally,
once the algorithm converges, the global LLR is obtained by multiply
the average local LLRs with the number of sensors in the network.
When the G-LLR is available, cloud declaration could be made based
on the ML or MAP decision rule.

To drive and simplify the expression of global log likelihood ratio
when sensor\textquoteright{}s observation which is described by the
received signal model in \ref{sub:Received-Signal'-Model}, let 
\[
\mathbf{x}=\left[x_{1},\ldots,x_{L}\right]^{\mathrm{T}}
\]


\[
\mathbf{n}_{m}=\left[n_{1,m},\ldots,n_{L,m}\right]^{\mathrm{T}}
\]


\begin{equation}
\mathbf{u}_{m}=\left[\mu_{1,m},\ldots,\mu_{L,m}\right]^{\mathrm{T}},\; m=0,1.
\end{equation}
 if $\mathbf{n}_{m}\sim\mathcal{N}\left(0,\Sigma_{m}\right)$, G-LLR
becomes, 

\begin{eqnarray}
LLR(\mathbf{x}) & = & -\frac{1}{2}\left(\mathbf{x}-\mathbf{u}_{1}\right)^{\mathrm{T}}\mathbf{\Sigma}_{1}^{-1}\left(\mathbf{x}-\mathbf{u}_{1}\right)+\frac{1}{2}\left(\mathbf{x}-\mathbf{u}_{0}\right)^{\mathrm{T}}\mathbf{\Sigma}_{0}^{-1}\left(\mathbf{x}-\mathbf{u}_{0}\right)+\frac{1}{2}log\left(\frac{\left|\Sigma_{0}\right|}{\left|\Sigma_{1}\right|}\right)\nonumber \\
 & = & \left(\mathbf{u}_{1}^{\mathrm{T}}\mathbf{\Sigma}_{1}^{-1}-\mathbf{u}_{0}^{\mathrm{T}}\mathbf{\Sigma}_{0}^{-1}\right)\mathbf{x}+\frac{1}{2}\left[log\left|\Sigma_{0}\right|-log\left|\Sigma_{1}\right|\right]\label{eq:G-LLR Expand}\\
 &  & -\frac{1}{2}\left(\mathbf{u}_{1}^{\mathrm{T}}\mathbf{\Sigma}_{1}^{-1}\mathbf{u}_{1}-\mathbf{u}_{0}^{\mathrm{T}}\mathbf{\Sigma}_{0}^{-1}\mathbf{u}_{0}\right)-\frac{1}{2}\left[\mathbf{x}^{\mathrm{T}}\left(\mathbf{\Sigma}_{1}^{-1}-\Sigma_{0}^{-1}\right)\mathbf{x}\right]\nonumber \\
 & = & \sum_{l=1}^{L}w_{l}x_{l}+C-\frac{1}{2}\left[\mathbf{x}^{\mathrm{T}}\left(\mathbf{\Sigma}_{1}^{-1}-\Sigma_{0}^{-1}\right)\mathbf{x}\right],\label{eq:G-LLR simple}
\end{eqnarray}
where $w_{l}$ denotes the $l\mbox{th}$ component of $\mathbf{u}_{1}^{\mathrm{T}}\mathbf{\Sigma}_{1}^{-1}-\mathbf{u}_{0}^{\mathrm{T}}\mathbf{\Sigma}_{0}^{-1}$,
and $C=\frac{1}{2}\left[log\left(\left|\Sigma_{0}\right|\right)-log(\left|\Sigma_{1}\right|)\right]-\frac{1}{2}\left(\mathbf{u}_{1}^{\mathrm{T}}\mathbf{\Sigma}_{1}^{-1}\mathbf{u}_{1}-\mathbf{u}_{0}^{\mathrm{T}}\mathbf{\Sigma}_{0}^{-1}\mathbf{u}_{0}\right)$. 

When no cloud exists, the sensor's observation signal is only caused
by atmospheric backscatter and noise. It is described by joint Gaussian
distribution $\mathcal{N}\left(\mathbf{u}_{0},\Sigma_{0}\right)$.
If these distributions are independent and identical, we have 
\begin{equation}
\mathbf{\Sigma}_{0}=\sigma_{0}^{2}\mathbf{I}.
\end{equation}
On the contrary, $\mathbf{\Sigma}_{1}$ can't be written in the same
form. Because the fluctuating amplitude of cloud concentration may
different from one sensor to another. In addition, fluctuating of
cloud concentration or sensor's observation are correlated, especially
for the sensors close to each other. As shown in Figure \ref{fig:GMM (x1,x2)},
the correlation is obvious when sensor's observation is shifted with
the right time delay. However, if the training data are available,
$\mathbf{\Sigma}_{1}$ and $\mathbf{u}_{1}$ can be obtained by expectation-maximization
(EM) algorithm. This is more related to sensor learning. In real sensing
area, sensors should need to learn their environments and get these
parameters updated and tracked. 

In a network without fusion center, it is desirable to find the $LLR(\mathbf{x})$
in a distributed manner by distributed consensus algorithm (DCA).
Some properties of DCA should emphasize here. First values can only
be exchanged between neighbors. Second, DCA is a tool to find the
average of the local values that initially hold by all the nodes in
the networks. 

The quadratic form in Eq.\ref{eq:G-LLR simple} contains high order
components of $\mathbf{x}$. On the other hand, without global information,
it seems not possible to calculate $\mathbf{\Sigma}_{1}^{-1}$ in
in a distributed manner. Both make it is impossible for DCA to calculate
G-LLR. So sensors can only build the Gaussian mixture model separately.
Thus, the observation of sensors is assume to be independent from
on to another. If we assume $\Sigma_{1}=diag\left(\sigma_{1,1}^{2},\sigma_{2,1}^{2},\ldots,\sigma_{L,1}^{2}\right)$,
$\Sigma_{0}=\sigma_{0}^{2}\mathbf{I}$, the Eq.\ref{eq:G-LLR Expand}
evolves into Eq.\ref{eq:Sum L-LLR Cloud}, in which the G-LLR is equal
to the average multiply with the number of sensors in the network. 

However, the assumption that correlation only exist between sensors
located very close to each other makes it possible to calculate the
G-LLR in Eq.\ref{eq:Sum L-LLR Cloud} using DCA. As each node only
stores the coefficient corresponding to itself. When signal are correlated,
we make approximation that $c_{ij}=0,$ if node $i$ and node $j$
are not neighbors. So in the term, 
\begin{equation}
\frac{1}{2}\sum_{i=1}^{L}\sum_{j=1}^{L}c_{ij}x_{i}x_{j}\label{eq:Term x_i*x_j}
\end{equation}
we can find the value of Eq.\ref{eq:Term x_i*x_j} by the following
algorithm:
\begin{enumerate}
\item Assume for all node $i$, it has the value $x_{i}$ and $c_{ij}$.
As the matrix is symmetric and the entries $c_{ij}=c_{ji}$. node
$i$ send $x_{i}$ and receive $x_{j}$ from all node $j$ in the
neighbors set ${\cal N}_{i}$. Compute the value 
\begin{equation}
v_{i}=\sum_{j\in{\cal N}_{i}}c_{ij}x_{i}x_{j}+c_{ii}x_{i}^{2}
\end{equation}

\item Initialize a DAC algorithm with local values $v_{i}$ until they converges
to the average $\bar{v}=\frac{1}{L}\sum_{i=1}^{L}v_{i}$. 
\item Start another DAC algorithm to find $\bar{u}=\sum_{l=1}^{L}w_{l}x_{l}$. 
\item The G-LLR is equal to $\bar{u}+\bar{v+C}$ multiply with the number
of sensors in the network.
\end{enumerate}

\subsection{Simulation\label{sec:Simulation}}

The distributed cloud detection simulation consists of two stages,
training and detection. In the training stage, sensors are trained
to build the Gaussian mixture models, which are important to calculate
the L-LLR. After the training, when a new observation comes, all sensors
calculated the L-LLRs and take them into DCA iteration to obtain G-LLR.
The decision of cloud existence could be made distributively once
the algorithm converges. 

The 3D cloud animation is simulated to generate a bunch of cloud plumes.
If we observing the cloud for a long time and take the average, the
result is Gaussian plume model. 

\begin{figure}
\hfill{}\includegraphics{\string"D:/Dropbox/PaperWork/CloudDetection/Cloud Detection Task Description/Mlti_Obs/SensorObs3\string".pdf}\hfill{}\hfill{}\caption{\label{fig:Sensor's obs}Sensor's observation impaired by Gaussian
white noise}
\end{figure}


The cloud animation is simulated several times to generate enough
data, which is divided into two group for both training and testing
of system performance. Because the turbulence flow is a random process,
the data generated each time is different with the others in the cloud
concentration distribution, as well as the cloud particles moving
track. This provides a very practical testing environment for the
system. 

Some other consideration for this simulation is that sensors are randomly
distributed in the sensing area and sensor's detections are impaired
by Gaussian white noise, which will be introduced in the following. 


\subsubsection{Sensor's Observation }

Gaussian white noise impaired the received signal when sensor observing
the cloud concentration. And the source of noise includes: (i) the
external noise, arising from the incidence of radiation at the detector
both from laser scattering and from the background; and (ii) the internal
noise, arising from fluctuations in the detector dark current and
thermal noise in the detector load resistor \cite{P.M.Hamilton1969}.
These noise is additive, and the overall noise is treated as a Gaussian
white noise denoted by $\mathcal{N}\left(u_{0},\sigma_{0}\right)$.
In this simulation, these parameters are chosen as $u_{0}=0.3$ and
$\sigma_{0}=0.1$, shown in Fig\ref{fig:Sensor's obs}.

Again, sensors are randomly distributed in the sensing area with the
uniform distribution. The sensing area in this simulation is defined
by pixels in the frame which has $x>D$, where $D$ is the distance
from the cloud source on the downwind direction. After they are distributed,
to get ready for the DCA algorithm, all the nodes will automatically
find its neighbors and obtain a network. Using DCA algorithm is to
obtain G-LLR without copy L-LLRs to all sensor nodes in the network.
Once the G-LLR is available, cloud declaration could be made base
on the ML or MAP decision rule.

Here we give an example, three sensors are distributed as shown in
Figure \ref{fig:Sensor's obs}. They build Gaussian mixture models
by processing the training data. Here we choose $K=1$. It can be
$2$ or more, depends on how many components of the noise in the environment.
Then, the testing data which generated by another cloud animation
is passing through all the sensors frame by frame. As shown in Figure
\ref{fig:Global-LLR-and}, L-LLRs for the two sensors and G-LLR is
given. Before frame 47, all the sensor has no contact with cloud,
and only noise are presented in each sensor. Only after sensor $S_{1}$
has its cloud contact, G-LLR raise to the first stage. After $S_{2}$
has its contact of cloud at frame 63, the G-LLR increased to an even
high level, which is strong evidence of the cloud existence. Because
$S_{3}$ had no chance to contact with the cloud, its L-LLR is actually
less sensitive to the sensors signal. Thus, its L-LLR has very low
contribution to the G-LLR. 

\begin{figure}
\hfill{}\includegraphics[width=14cm]{\string"D:/Dropbox/PaperWork/CloudDetection/Cloud Detection Task Description/Mlti_Obs/Sensor_LLR_Obs\string".pdf}\hfill{}\hfill{}\caption{\label{fig:Global-LLR-and}(upper) Global LLR vs. Local LLR and (down)
Sensor Observation (Only frame 40 to 80 is shown)}
\end{figure}


Another interesting thing is that, if two sensors $S1$ and $S2$
are put close to each other, especially when they are very nearly
located on the cloud particle's moving path, their observation have
high correlation. For example in Figure \ref{fig:Sensor's obs} and
Figure \ref{fig:GMM (x1,x2)}. The time delay $\tau$ is simply calculated
by local wind velocity and sensor\textquoteright{}s distance. The
cross-correlation can be another feature of the moving cloud.

\begin{figure}
\hfill{}\subfloat[{\label{fig:GMM (x1,x2)}The distribution of tuples $[x_{1}(t),x_{2}(t-\tau)]$,
$\tau$ is the time delay }]{\includegraphics[width=7cm,height=6cm]{\string"D:/Dropbox/PaperWork/CloudDetection/Cloud Detection Task Description/Mlti_Obs/Two Sensors' Correlation\string".pdf}}\hfill{}\subfloat[\label{fig:Two-Obs.}Two sensors observation $x_{1}(t),x_{2}(t)$]{\includegraphics[width=7cm,height=6cm]{\string"D:/Dropbox/PaperWork/CloudDetection/Cloud Detection Task Description/Mlti_Obs/Two Sensors' Obs\string".pdf}

}\hfill{}\caption{Sensor's observation of background noise and cloud backscatters }
\end{figure}



\subsubsection{Performance of Cloud Detection Sensor Network}

\begin{figure}
\hfill{}\includegraphics[width=9cm,height=7.5cm]{\string"D:/Dropbox/PaperWork/CloudDetection/Cloud Detection Task Description/ROC/ROC_mix_RandPos_256x192\string".pdf}\hfill{}\hfill{}\caption{\label{fig:ROC-of-CDWSN}Detection relative operating characteristic
for different sensors numbers}
\end{figure}


Simulation is running for hundreds of times to give the average performance
for different number of sensors in the network. The sensor's positions
are chosen randomly in the area where $x>128$, The Figure \ref{fig:ROC-of-CDWSN}
gives the relative operating characteristic (Also known as a ROC curve)
of the detection system with different sensors numbers. The curve
is represented by plotting the fraction of true detection out of the
cloud exist vs. the fraction of false alarm out of the cloud not exist.
When there is only one sensor is operation, a special case is that
the position is chosen by hand to make sure the sensor can contact
with the cloud plume. With this assurance, the one sensor detecting
system could have performance close to the three sensor detecting
system randomly distributed. The advantage of using distributed detection
with multiple sensors is obvious. As the detecting system with more
sensors is more reliable to noise which leads to a higher performance. 


\subsection{Conclusion}

The simulation of multiple observation shows that the detections of
sensors are correlated when they are located in short distance. The
correlation can be an indicate of the cloud plume. Because Gaussian
noise or Moving object cause very low correlation between sensors.
If this can be recognized, it will not likely to raise the false alarm.
In the noisy environment, sensors will adopt the expectation-maximization
algorithm to build the model of background noise and cloud backscatter.
By assume that sensors observation is independent, the global log
likelihood ratio is the average of local log likelihood ratio. Therefore,
global log likelihood ratio can be calculated by distributed consensus
algorithm, where each sensor take local log likelihood ratio into
the DCA iteration. Once the algorithm converges, the global log likelihood
ratio is available to each sensor. The cloud deceleration could be
made distributively based on the ML or MAP decision rule. 


\subsection{Further Research}

In the future research, the parameters of the plume are treated as
random. Therefore, sensors should be able to estimate these parameters
from their observation. Gaussian plume model can only describe the
mean value of a cloud plume. So a modified Gaussian plume model should
be developed. The correlation can be an indicate of the cloud plume
and will be considered in further research to improve the performance.
