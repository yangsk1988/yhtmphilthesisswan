
\chapter{\label{sec:Conclusion-and-Future}Conclusion and Future Work}

This section gives a summary of this thesis and direction for future
research.


\section{Conclusion}

In this thesis, at first asymptotic and non-asymptotic DAC algorithms
are reviewed. The asymptotic algorithms are robust against topology
changes and their optimization  needs to know the network topology,
which is very diffcult to obtain for any individual node and can not
change during the optimization. Non-asymptotic algorithms can find
the average faster in a time-invariant network.They use a finite impulse
response filter to estimate the consensus value. However, the filter
estimation is not reliable and will be interrupted once the topology
is changed, because outdated information will lead to a wrong answer
of the filter. Therefore, to choose the suitable DAC algorithm, it
depends on the network properties. 

 

Second, in section \ref{sub:Finite-time-Consensus-on}, a generalized
finite-time DAC algorithm is presented. Compared to the previous version
of finite-time DAC, the number of iteration can be reduced to $2n$,
where $n$ is the number of nodes in the network. In addition, the
 eigenvalues of the associated weight matrix are not required prior
to the algorithm. Actully, the number of iterations can be further
reduced to the number of distinct eigenvalues of the associated weight
matrix. 



Third, in chapter \ref{sec:Online-Optimization-of}, we proposed a
distributed real-time optimization method to increase the convergence
rate of asymptotic DAC algorithms. As stated in chapter \ref{sec:Online-Optimization-of},
the optimal parameters is only related to eigenvalues of the Laplacian
matrix associated to the network. Therefore, the key of the optimization
algorithm is to distributively estimate the eigenvalues.  However,
numerical errors of these parameters due to quantization can decline
the algorithm performance. To mitigate this effect, we introduce a
numerical technique to find the least mean square solution. If floating
point number in double format is used and the network size is smaller
than 32 nodes, the numerical errors of estimated parameters after
mitigation will only slightly declines the performance of higher-order
DAC.  Otherwise, the numerical errors will be too large even after
mitigation.  Generally, optimized parameters for the old network are not
optimal or even not located in the convergence region for the new
network. So the DAC algorithm must be reinitialized with convergent
parameters to continue the consensus process, until the network topology
is stable for a certain time and the optimal parameters are estimated. 

Finally, we introduced a distributed detection of cloud plume using
wireless sensor networks and the DAC algorithm in chapter \ref{sec:DAC-Implementation:-Distributed}.
First, the hypothesis testing based on the ML or MAP decision rule
is introduced. In the outdoor environment, signal of an individual
sensor might be corrupted by Gaussian noise or moving object with
high reflect coefficient, which might raise the false alarm. Because
multiple sensors detection have better performance, it was adopted
in the cloud detection. Expectation-maximization algorithm is used
here to build the joint Gaussian model of background noise and cloud
backscatter. Thus, interferences to a few sensors in the network are
less likely to raise the false alarm. Second, if we assume sensors
signals are independent, the global log likelihood ratio in the hypothesis
testing can be calculated by the DAC algorithm. , 


\section{Future Work}

In this thesis, we introduced several DAC algorithms.  It seems to
be impossible to find an algorithm robust against topology changes
and faster than finite-time DAC in a distributed and dynamic network.
It is very demanding for any individual node to know the network topology. 

Therefore, further research could be carried out to optimize the existing
algorithms and make some modifications according to the applications.

First, the distributed real-time optimization to DAC can be applied
on a distributed system. Before that, a consensus protocol should
be develop and implemented. We are supposed to deal with some of the
problems in practice, such as link failure, channel noise, time-delay
and asynchronous communication. 



Second, the distributed real-time optimization to DAC might be able
to applied in a dynamic network.  As stated in chapter \ref{sec:Online-Optimization-of},
the optimal parameters is only related to eigenvalues of the Laplacian
matrix associated to the network. Therefore, optimized parameters
for the old network are usually not optimal or not located in the
convergence region for the new network. However, simulation result
shows that constant first-order DAC algorithm and second-order DAC
algorithm can maintain the convergence in most of the time, even old
parameters are used. Some research should be carried out to find out
the conditions to maintain the convergence. 

Third, to avoid complex analysis, the network graph is assumed to
be symmetric in the analysis. However, some algorithm, such as the
finite-time DAC algorithm and the eigenvalue estimation algorithm,
can be generalized to the case when the network in unsymmetrical.
Therefore, the distributed real-time optimization might able to be
generalized the unsymmetrical network. 

Finally, for the application of cloud detection, the cloud model is
going to be extended to capture the correlation and random properties
of the cloud plume. As a result, some parameters of cloud plume will
be treated as random variable, the DAC algorithm need to be modified
to capture these changes.  At the same time, sensors need to be able
to estimate these parameters from their observation. 
