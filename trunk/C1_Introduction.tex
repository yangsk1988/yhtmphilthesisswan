
\chapter{Introduction }



































Due to the development of wireless sensor networks and the decreasing
cost of sensor nodes,  processing information that originally  acquired
by the nodes is often necessary. As any node only has a piece of local
information,  all information need to be gathered and processed at
a special node called  fusion center. This method is intuitive  and
is called centralized signal processing. 

However, fusion center is usually the most important and expense node
in the network. Once it is destroyed or failed, the whole network
breaks down. In addition, as data gathering is an energy consuming
process, centralized signal processing is limited due to energy constrain
of mobile nodes. Moreover, there is unbalance energy consumption between
nodes with different communication loads \cite{Cristescu2004}\cite{Yuen2008}. 

Distributed signal processing has become more attractive, as it can
be robust against  nodes failure or topology changes \cite{Chair1986}
. However, not all the signal processing can be distributed many processing
algorithm are still implemented by centralized method.

Distributed average consensus (DAC) algorithm is a tool for distributed
information processing. It has received significant attention recently
because of its robustness and simplicity. It is widely used in many
applications such as time synchronization \cite{Schenato2011}, rendezvous
\cite{Ren2008}, cooperative control of vehicles \cite{Yang2010},
formation control/flocking \cite{Olfati-Saber2012} and WSNs \cite{Hlinka2012}.
In these applications, it is often necessary that a group of nodes
can agree on certain quantities. 



In different applications, DAC algorithms may also need a little modification.
For example, in distributed tracking of a moving target by wireless
sensor networks. Suppose each sensor is observing the target coordinates
but the output is corrupted by independent and identically distributed
zero-mean Gaussian noise, to minimize the interference from the noise,
the sensors need to take the average of all initial values. In distributed
hypothesis testing, the goal is to find the production of the global
likelihood ratio which is the product of the local likelihood ratio.
Therefore, each sensor first calculate the log of the local likelihood
ratio and substitute it into the DAC algorithm. The average obtained
by DAC algorithm multiplied with the number of nodes in the network
is log of global likelihood ratio.

In practice, DAC algorithms also face the challenges of link failure,
time delay, dynamic network, asynchronous communication and other
practical aspects. Therefore, a simple but reliable solution sometimes
performs better. First-order DAC algorithms have been proved to be
robust against  topology changes and they play important roles in
practice \cite{Ren2007}. Its optimization in a dynamic network still
attract lots of interest \cite{Jakovetic2010}\cite{Nedic2010}. 


\section{Motivation of Research }

The research is motivated by the distributed detection of cloud detection
using WSN. One of the properties of mobile sensor nodes is the limited
energy capacity. Therefore, to minimize the power consumption of the
sensor nodes, DAC algorithm to perform the distributed signal processing
need to be optimized. In addition, the model of cloud requires a modification
because the sensor is using laser sensing technology. It emit a laser
to illuminate the cloud and collect the backscatters to sense the
cloud concentration.


\subsection{To Find a Faster DAC Algorithm. }

  

In the distributed tracking of a moving target, when the moving target
is highly dynamic or the sensors need to sample at a very high frequency,
it requires that the DAC algorithm returns the result in a short time.
Thus, many efforts have been devoted to optimize the algorithm. 

DAC algorithms can be divided into asymptotic and non-asymptotic algorithms.
For asymptotic algorithms, the optimization is to minimize the sub-dominate
eigenvalue of a weight matrix \cite{Asensio-Marco2012}\cite{Xiao2004}\cite{Xiong2010}.
 However, these optimization of DAC algorithms are centralized methods
. 

For non-asymptotic DAC algorithms, such as finite-time \cite{Sundaram2007}
and adaptive filter DAC algorithm \cite{Cavalcante2010}, the optimization
is to minimize the number of necessary iteration before a FIR filter
is estimated. Sumdaram and Hadjicostis \cite{Sundaram2007} verify
that there exists a FIR filter that can estimate the consensus value.
Cavalcante and Mulgrew \cite{Cavalcante2010} follow Sundaram and
Hadjicostis's work to propose an adaptive algorithm to find the filter.
However, they are not robust against topology changes.   Local values
over time obtained by the first-order DAC are taken as inputs of the
filter estimation algorithm for both of them. As a result, if the
network topology changes, these algorithms have to be reinitialized,
as outdated information during the filter estimation will lead to
a wrong answer.

To enable the whole system work distributively, the DAC algorithms
should be robust against topology changes and the optimization should
be distributed. 

A distributed method inspired by the gossip algorithm \cite{Boyd2006}
can be used to optimize the first order DAC but it converges very
slowly. The method involves triple nested distributed matrix iterations.
The inner iteration has to converge to a certain range so that the
iteration outside can return the right result. Thus, It is not surprising
that it could not finish in a reasonable time when the network size
is large. 

Therefore, a distributed optimization method with less computation
time is required. In addition, it is better that no additional communication
cost is required. Moreover, if the optimization algorithm and the
DAC algorithm can be executed simultaneously, then consensus process
will not be interrupted and the optimization can be running in background
to keep the optimal parameters updated in a dynamic network. 


\subsection{To Detect The Cloud Plume by Laser Sensor}

The cloud here is a group of harmful gas or particles floating in
the air. To detect the cloud, it is illuminated by a laser beam emitted
by a sensor and the backscatters (light reflected by particles) are
collected to generate signals. 

Because the laser beam penetrate the cloud, the intensity of backscatters
is the integration along the laser beam in that range. Consequently,
a new cloud plume model need to be obtained by take integration of
original Gaussian plume model along the direction of laser beam. 

Specifically, in the cloud detection application, the DAC algorithm
performs the task of distributed hypothesis testing may also have
a little modification. To find the production of the global likelihood
ratio, each sensor first calculate the log of the local likelihood
ratio and substitute it into the DAC algorithm as the initial local
value. DAC algorithm will then find the average. The average multiplied
by the number of nodes in the network is log of global likelihood
ratio.


\section{Contributions}

The main contribution of this thesis is that it proposed a distributed
real-time optimization method, which will not interrupt the constant
first-order DAC algorithm. Thus, the consensus algorithm and the optimization
method can be executed simultaneously. As a result, communication
cost and optimization initialization time can be dramatically reduced
compare to conventional distributed optimization. In addition, a least
mean square solution is obtained to mitigate the numerical error of
the optimal solution calculated by the proposed method. When using
floating point number in double format and the network size is smaller
than 32, the numerical error after mitigation does not dramatically
decline the algorithm performance. 

In addition, the proposed distributed eigenvalue estimation algorithm
could be an alternative to the estimation of algebra connectivity,
such as \cite{Yang2010}. The proposed method could obtain the result
with sufficient numerical accuracy. Matrix iterations only need to
be executed for times in order of $n$. 

Moreover, we propose a generalized finite-time DAC algorithm, which
does not require knowledge of network topology but will estimate the
necessary parameters during the iteration. Compared to the distributed
algorithm introduced in \cite{Sundaram2007}, it doesn't require the
re-initialization of the constant first-order DAC algorithm for several
times. Therefore, total data transmission in all iterations can be
reduced. However, re-initialization can be introduced if the consensus
value requires to have high accuracy 

Finally, a modified Gaussian plume models is presented, which is the
integration of the original Gaussian plume model along the laser beam
to simulate the laser penetration. This Gaussian plume model can only
describe the mean of the cloud concentration. However, a real cloud
cloud concentration is actually a random process. To reveal  the dynamics
and turbulence properties of the cloud plume, a 3D cloud animation
is implemented with the computer graphic technology \cite{He2011}. A
bunch of the simulated cloud plumes is generated for algorithm testing
and detection.




\section{Outline}

This thesis is structured as follows: Chapter  \ref{sec:Consensus-problem-on}
is the review of some asymptotic and non-asymptotic DAC algorithms,
as well as some distributed signal processing methods based on DAC
algorithm. In chapter  \ref{sub:Finite-time-Consensus-on}, a generalized
finite-time DAC algorithm will be presented. Compared to the previous
version of finite-time DAC, the number of iteration can be reduced.
In chapter \ref{sec:Online-Optimization-of}, a distributed real-time
optimization method to increase the convergence rate of asymptotic
DAC algorithms. Finally, a distributed detection of cloud plume using
wireless sensor networks and the DAC algorithm will be introduced
in chapter \ref{sec:DAC-Implementation:-Distributed}. Chapter \ref{sec:Conclusion-and-Future}
concludes this thesis and gives out the direction of further work.
