
\title{Consensus Based Signal Processing and its Application in Cloud Detection}
\maketitle
\begin{abstract}
Distributed Consensus algorithm (DCA) has application in data fusion
in wireless sensor networks and coordination of multi-agent system.
Some distributive signal processing based on consensus algorithm ,
such as, local value linear prediction, information flooding and network
eigenvalues estimation, were introduced in this thesis. Finally, a
simulation of distributive cloud detection was given, in which data
acquired by sensor nodes is processed by distributed consensus algorithm.
A modification to the conventional consensus algorithm is also applied
to make it possible to find the global log likelihood ratio when signals
are correlated.
\end{abstract}

\section{Introduction }

In a distributed system, processing information that originally and
locally acquired by the nodes is often necessary. Centralized signal
processing require a special node called fusion center. Sometimes
fusion center has to be involved if the signal processing is complicate
and it is not possible to find a distributive solution. However, fusion
center is the most important node in the network that responsible
for data gathering and data processing. Once the fusion center is
destroyed or failed, the whole network breaks down. In addition, limited
energy is also a problem for WSN in centralized signal processing
because data gathering is a energy consuming process. Also there is
unbalance energy consumption between nodes with different communication
loads \cite{Cristescu2004}\cite{Yuen2008}. In contrast, distributed
signal processing involves multiple sensors date fusion, can normally
be more reliable than single sensor node detection \cite{Chair1986}\cite{Kam1992}.
Also the distributed signal processing has the advantages of reliability
to nodes failure or topology changes as well as balanced energy consumption.
 

Distributed Average Consensus (DAC) algorithm has received significant
attention recently because of its robustness and simplicity for processing
distributed information over networks. It have applications in rendezvous,
formation control, flocking, attitude alignment and wireless sensor
networks \cite{Ren2007}. For these applications, reaching to an agreement
among all the nodes in a distributed network is often necessary. There
are many algorithms developed by now. We discuss these DAC algorithms
by mainly comparing the convergence rate of them. The agreement is
achieved if local values of nodes converge to a common value which
is referred to as a consensus value. 

In conventional DAC algorithms, each node updates its local value
repeatedly as a weighted linear combination of its previous local
value and those from its neighbors. The iteration is continued until
all the local values converge to a global average within a desired
error level. These methods are referred as asymptotic consensus in
\cite{Sundaram2007}. By introducing the associated network graph
matrix, whose entries are the corresponding weights assigned to the
edges, the convergence rate of linear DAC is determined by the spectral
radius of the matrix. Commonly, the edge weights are determined by
degree of nodes and their adjacent node's degrees so that the DAC
algorithms would converge \cite{Xiong2009a}. 

To find an algorithm that finds consensus value more quickly, a great
deal of research interest has been devoted to more sophisticated consensus
algorithms \cite{Sundaram2007}, \cite{Xiao2004}, \cite{Moreau2004},
\cite{Khan2010}, \cite{Li2010}, \cite{Kokiopoulou2007}, \cite{Cavalcante2010}.
For the first order linear consensus algorithm, convergence rate is
increased by casting graph matrix spectral radius minimization problem
as a convex optimization problem \cite{Xiao2004}, but solving such
problem requires knowledge of the whole network topology. Second order
or high-order DAC can be introduced to improve the convergence rate.
It also requires knowledge of network topology but only eigenvalues
of Laplacian matrix \cite{Xiong2009a}, which only depends on the
network graph and is very useful tool to analyze the convergence rate.
Because of the properties of the Laplacian matrix, its first smallest
eigenvalue is always zero and the second smallest eigenvalue (called
the algebraic connectivity of the graph, or denoted as the Fiedler
eigenvalue) is equal to zero if and only if the graph is not connected
\cite{MiroslavFiedler1973}\cite{Russell1994}. Theoretically, higher
order algorithm will potentially have a higher convergence rate.\textbf{
}However, there are very little improvement when introducing the order
larger than fourth \cite{Xiong2010}. In \cite{Moreau2004}, the consensus
problem is treated in a continuous-time manner and the local value
is described by a set of differential equations. \cite{Khan2010}
considers higher dimensional consensus (HDC), which is a general class
of linear distributed algorithms for large-scale networks. In HDC,
the network nodes are partitioned into \textquotedblleft{}anchors\textquotedblright{},
whose states are fixed over the iterations, and \textquotedblleft{}sensors\textquotedblright{},
whose states are updated by the algorithm. And \cite{Li2010} proposed
a class of location-aided distributed averaging algorithms which accelerate
the slow convergence by overcoming the diffusive behavior of reversible
chains. 

It needs mentioned that some novel methods increase the convergence
rate by taking a number of consecutive local values obtained by asymptotic
consensus into consideration. Motivated by the Kalman filter scheme,
a consensus update scheme is proposed in \cite{Ren2005}, which treats
the final consensus value as the system state, and it is shown that
algorithm is convergent for strongly connected networks. Later, a
modification to the basic Kalman filter algorithm is presented to
ensure the Kalman Filter converges to an unbiased estimate \cite{Alighanbari2006}.
The method in \cite{Kokiopoulou2007} applies scalar epsilon algorithm
(SEA) to the local values sequence in each node to accelerate the
convergence. The main disadvantage of this method is that it uses
all previous estimates and its robustness against changes in the network
topology is questionable \cite{Cavalcante2010}. In turn, Cavalcante
and Mulgrew \cite{Cavalcante2010} introduces an adaptive filter algorithm
to find the consensus value by iteratively updating the coefficients
of an adaptive filter. Another novel method \cite{Sundaram2007} tries
to find the z-transform of the nodes value, and uses final value theorem
to obtain the consensus value. However, every node having the knowledge
of network topology is a strict condition. Therefore, a decentralized
method is proposed to computer the matrix polynomial. This method
requires several runs of the consensus algorithm initialized with
a set of linear independent vectors.

Motivated by the works in \cite{Kokiopoulou2007}, we try to rewrite
the local value vectors in another form which reveals an important
property of the algorithm. Based on this fact, we propose some signal
processing techniques. For example, with the proposed consensus finding
filter, each node could estimate the consensus value by passing through
consecutive local values obtained by the first order linear DAC algorithm
without more information exchanged between nodes.


\subsection{Categories of Consensus Algorithm}

To introduce the distributive consensus algorithm, we may started
from one of the simplest algorithm in their family. \prettyref{fig:Categories-of-Discrete-time}
only shows the branch of discrete-time distributive consensus algorithm,
continuous-time consensus algorithm has some similarities with discrete-time
ones. However, their performance is quite different when Gaussian
white noise exists, so it can be categorized in another branch. As
shown in \prettyref{sub:DT-First-Order-DAC}, the first order DAC
algorithm is the simplest algorithm that can solve the consensus problem
in a number of iteration. It's convergence rate is related to the
spectral radius of a graph dependent matrix. So the optimization problem
is to find the optimal matrix with minimum spectral radius. However,
global information of the graph matrix must be available. In distributed
methods, this is a quite strong condition. Without the global information,
best constant and metropolitan matrix can be the sub-optimal solution
to the consensus problem \cite{Xiao2004}. The first order DAC algorithm
is one of the asymptotic algorithms together with higher-order DAC
algorithm. The demand of higher-order algorithm is mainly because
the requirement for a fast convergence rate is always a matter of
issue. In \prettyref{sub:Discrete-High-Order}, we introduced higher-order
DAC algorithm, which could have faster convergence rate, and no additional
requirement and communication cost is required compared with first-order
DAC \cite{Xiong2010}. Therefore, it is shortly applied into practical
consensus problem after invented. (OK first half 0.5)

Some of the novel methods can solve the average consensus problem
in finite number of iterations. These methods are referred to as finite-time
consensus algorithm \cite{Sundaram2007}. It actually a very sophisticated
signal processing technique that find the asymptotically stable equilibrium
$\bar{x}$ by extrapolation, see \prettyref{sec:Finite-Time-Distributed-Consensu}.
A derivative of finite-time consensus algorithm is the adaptive filter
algorithm for average consensus . It applies an adaptive filter algorithm
to asymptotically converge the set of parameters which are required
in the computation of consensus value \cite{Cavalcante2010}. Reader
can find the definition of these parameters in \prettyref{sub:Find-the-Consensus}.
As a contribution of this thesis, a method to calculate these necessary
parameters by inverting Toeplitz matrix is introduced in \prettyref{sub:Finite-time-Consensus-on}. 

\begin{figure}
\hfill{}\includegraphics[width=13.5cm]{\string"Graph/Categories of DAC_Nor\string".pdf}\hfill{}

\caption{\label{fig:Categories-of-Discrete-time}Categories of Discrete-time
Consensus Algorithm}
\end{figure}

