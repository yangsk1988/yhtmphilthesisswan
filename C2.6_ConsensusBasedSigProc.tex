
\section{Consensus Based Signal Processing}

Sensor networks have a variety of applications such as surveillance,
environment monitoring and collaborative signal processing. As the
advatages of reliability, survivability, and increased range of coverage,
there is an increasing interest in employing multiple distributed
sensors for these applications \cite{Chair1986}. A fundamental problem
in sensor network is to process spatially distributed information
using a scalable algorithm \cite{Olfati-Saber2005a}. 

Generally, there are two options for multiple sensors signal processing:
First option is centralized signal processing. This requires the network
contains a fusion center, and all sensor's information being transmuted
to the central processor. 

The Second option is distributed signal processing. Distributed average
consensus (DAC) algorithm is a tool for distributed information processing.
It has received significant attention recently because of its robustness
and simplicity. In this section, we will introduce two distributed
signal processing methods based on consensus algorithm.


\subsection{Data Fusion and Decision Making(todo check times 2)}

In this section, we will consider a distributed detection problem
in wireless sensor network without the fusion center. A consensus
based approach of distributed data fusion and decision making is introduced.

In centralized data fusion and decision making, a hypothesis test
based on the ML, MAP or Bayesian decision rule will be carried out
at the a fusion center.  Therefore, we intend to carry out the hypothesis
test in a distributed manner. 

Considering a binary hypothesis testing problem with the following
two hypotheses
\begin{enumerate}
\item H0: target is absent
\item H1: target is present.
\end{enumerate}
Suppose each sensor acquires a scale value from its sensoring area,
its signal has the following form

\begin{equation}
x_{l}=\begin{cases}
\mu_{l,0}+n_{l} & \mbox{if no target presents}\\
\mu_{l,1}+n_{l} & \mbox{if target presents}
\end{cases}
\end{equation}
where $\mu_{l,m}$ is the mean value of $x_{l}$ depending on hypothesis
$m$ and $n_{l}$ is the noise of $x_{l}$. The prior probability
of these two hypotheses is denoted by $P\left(H_{m}\right)=P_{m},m=1,2$.
The case when each sensor acquires a vector of unknown parameters
is discussed in \cite{Xiao2005}, where a more sophisticated data
fusion scheme is proposed.   

To make a declaration whether the target presents or not, based on
classical hypothesis test theory, the global log likelihood ratio
(G-LLR ) test is given by 

\begin{equation}
LLR(x_{1},...,x_{L})=\log\frac{f\left(x_{1},...,x_{L}|H_{1}\right)}{f\left(x_{1},...,x_{L}|H_{0}\right)}\underset{H_{0}}{\overset{H_{1}}{\gtrless}\log}\frac{P\left(H_{o}\right)}{P\left(H_{1}\right)}\label{eq:G-LLR define}
\end{equation}
where $f\left(x_{1},...,x_{L}|H_{m}\right)$ is the likelihood function
of hypothesis $H_{m}$.

Usually  we can assume the sensor detections are independent from
one to another. Therefore, we have $f\left(x_{1},...,x_{L}|H_{m}\right)=f\left(x_{1}|H_{1}\right)\cdot\ldots\cdot f\left(x_{L}|H_{1}\right)$
and 

\begin{eqnarray}
LLR(\mathbf{x}) & = & \log\frac{f\left(x_{1}|H_{1}\right)\cdot\ldots\cdot f\left(x_{L}|H_{1}\right)}{f\left(x_{1}|H_{0}\right)\cdot\ldots\cdot f\left(x_{L}|H_{0}\right)}=\sum_{i=1}^{L}LLR\left(x_{i}\right)\label{eq:Sum_L_LLR}
\end{eqnarray}


According to Eq.\eqref{Sum_L_LLR}, the G-LLR is the sum of local
log likelihood ratio (L-LLR) and can be calculated by distributed
average consensus (DAC) algorithm. This is implemented by the following
steps: First, each sensor node $i$ only calculate its L-LLR individually
based on $x_{i}$; Then, all the sensors update their L-LLR in the
DAC iteration until they converge to a common value; Finally, once
the algorithm converges, the G-LLR is obtained by multiply the average
value with the number of sensors in the network.


\subsubsection*{Generalization of LLR Calculation }

In this section, we will generalize the conditions step by step and
drive some expressions to show how to calculate G-LLR with DAC algorithms. 

First we assume the noises in sensor's signals are not independent
from one to another. The noises can be denoted by the joint Gaussian
white noise. 

Let the sensors observation, the joint Gaussian white noises and the
mean of sensor observation in a vector form given by 
\[
\mathbf{x}=\left[x_{1},\ldots,x_{L}\right]^{\mathrm{T}}
\]


\[
\mathbf{n}=\left[n_{1},\ldots,n_{L}\right]^{\mathrm{T}}\sim\mathcal{N}\left(0,\Sigma\right)
\]


\begin{equation}
\mathbf{u}_{m}=\left[\mu_{1,m},\ldots,\mu_{L,m}\right]^{\mathrm{T}},\; m=0,1.
\end{equation}
Since $\mathbf{n}\sim\mathcal{N}\left(0,\Sigma\right)$, the likelihood
function is a joint Gaussian function $f\left(x_{1},...,x_{L}|H_{m}\right)=\frac{1}{\left(2\pi\right)^{L/2}\left|\Sigma\right|^{\frac{1}{2}}}\exp\left(-\frac{1}{2}\left(\mathbf{x}-\mathbf{u}_{m}\right)^{T}\Sigma^{-1}\left(\mathbf{x}-\mathbf{u}_{m}\right)\right)$,
the G-LLR becomes 
\begin{eqnarray}
LLR(\mathbf{x}) & = & \left(\mathbf{u}_{1}^{\mathrm{T}}-\mathbf{u}_{0}^{\mathrm{T}}\right)\mathbf{\Sigma}^{-1}\mathbf{x}+\frac{1}{2}\left(\mathbf{u}_{0}^{\mathrm{T}}\mathbf{\Sigma}^{-1}\mathbf{u}_{0}-\mathbf{u}_{1}^{\mathrm{T}}\mathbf{\Sigma}^{-1}\mathbf{u}_{1}\right)\label{eq:wighted_sum_LLR}\\
 & = & \sum_{l=1}^{L}w_{l}x_{l}+C\nonumber 
\end{eqnarray}
where $w_{l}$ is the $l^{th}$ component of $\left(\mathbf{u}_{1}^{\mathrm{T}}-\mathbf{u}_{0}^{\mathrm{T}}\right)\mathbf{\Sigma}^{-1}$,
$C$ is the last term the equation. Eq. \eqref{wighted_sum_LLR} means
the G-LLR can be a weighted sum of sensor's observation. 

Provided that each sensor knows the weight $w_{l}$ and $C$, the
G-LLR is equal to the weighted sum of sensor's observation plus $C$.
Actually, the constant $C$ changes the threshold of the hypothesis
testing and can be subtract from both side of the equation. Therefore,
we modify the hypothesis testing into 
\[
\sum_{l=1}^{L}w_{l}x_{l}\underset{H_{0}}{\overset{H_{1}}{\gtrless}}\frac{P\left(H_{o}\right)}{P\left(H_{1}\right)}-C
\]
where the average values $\mbox{Avg}\left(w_{l}x_{l}\right)$ is obtained
by distributed average consensus (DAC) algorithm. Then the average
is multiplied with the number of sensors in the network to get $\sum_{l=1}^{L}w_{l}x_{l}$.


In the section of cloud detection \secref{Distributed_Cloud_De},
more generalized conditions of calculating the G-LLR is discussed
and more results about applying DAC algorithm will be presented. The
sensor signals will be correlated and mixed with joint Gaussian white
noises. In addition, the noises are not only correlated but also depends
on the existence of target. 


\subsection{Network Information Flooding(todo check times 0)}

The conventional information flooding is actually done by copying
information to all other nodes. Each node maintains a table of the
of all nodes values in the network, initialized with its own value,
and exchanges the tables of their own with those from their neighbors
in each step. After a certain number of time steps which is equal
to the diameter of the network, each node will obtain values of all
nodes. distributed averaging can also be implemented by this way. 

However, copying information and forwarding to all other nodes takes
too much resources for the networks. If a networks has $n$ nodes,
the conventional flooding will needs at least $\left(n-1\right)$
copies for each piece of information. And this estimation doesn't
considering the cost in transmitting the information to the destination.

We intend to develop an algorithm to accomplish the information flooding,
but didn't copy all the information to other nodes, so that the communication
cost can be dramatically reduced. However, The difficulties of this
kinds of distributed signal processing is that any individual node
only know partial information on the whole network. And information
exchange only allowed between neighbors. 

Based on the FT-DAC algorithm we introduced in the section \ref{sub:Finite-time-Consensus-on},
each node could decompose the local value vector by a linear combination
of eigenvalues and eigenvectors and can calculate the coefficients
before each term. Actually, with some signal processing techniques,
there are more information in the local values sequence that each
node can extract. These may also be applied to wireless sensor networks
as they could be carried out distributively. 

Therefore, in this section we propose a novel network information
flooding technique based on consensus algorithm. It doesn't require
copying information for so many time, instead, it transmit information
by broadcasting. The advantage of this method is that it can save
the costs in copying and transmitting information. 

Suppose the network weight matrix $W$ satisfies the same condition
\ref{eq:convege condition W}. Recall the initial local value decomposition
given by

\begin{equation}
\mathbf{x}\left(0\right)=\sum_{j=1}^{n}\alpha_{j}\mathbf{e}_{j}\label{eq:initial vector decompose-1}
\end{equation}
which shows that the initial local value vector can be defined by
a set of coefficients and a new basis defined by the eigenvectors.\textbf{
}In addition, the eigenvectors $\mathbf{e}_{i}$ are only depends
on the weight matrix. Therefore, once the coefficients $\alpha_{i}$
and weight matrix are available, the initial values vector $\mathbf{x}\left(0\right)$
can be calculated. It is possible to exchange information between
nodes in the network without copying information to all other nodes.

To show how this method can be carried out distributively, let the
sample vector $\mathbf{y}_{i}(k,d)\in R^{d}$ is defined by the history
of $x_{i}(k)$ 

\begin{equation}
\mathbf{y}_{i}(k,d)=\left[x_{i}(k),x_{i}(k-1),\ldots x_{i}(k-d+1)\right]^{\mathrm{T}}
\end{equation}
and the coefficients vector $\mathbf{a}=\left[\alpha_{1},\alpha_{2},\ldots,\alpha_{n}\right]^{T}$.
If we involve the eigenvectors and redefine the \eqref{Vander_beta}
by 
\begin{equation}
\mathbf{y}_{i}(k,d)=\mathbf{V}_{i}(k,d)diag\left(\left[\mathbf{e}_{1}^{T}\mathbf{u}_{i},\mathbf{e}_{2}^{T}\mathbf{u}_{i},\ldots,\mathbf{e}_{n}^{T}\mathbf{u}_{i}\right]\right)\mathbf{a}\label{eq:Vander_Eigen_ith_Alpha}
\end{equation}
where $\mathbf{u}_{i}$ is the unit vector with all zero except the
$i^{th}$ component is one, $\mathbf{e}_{j}^{T}\mathbf{u}_{i}$ means
the $i^{th}$ component of $\mathbf{e}_{j}$. Solving the above equation
will obtain the coefficients $\alpha_{j}$. 

The consensus based information flooding is ideally suitable for time-invariant
network. Because this method requires a node know all the eigenvectors
and eigenvalues of the network before it can estimate initial values
of other nodes. It can be initialized by all flooding weight coefficients
to all nodes in the network. However, instead of flooding a table
of local values, flooding the weight matrix is only performed at the
stage initialization for one time. The proposed method could have
lower cost both in computation and communication, if the topology
changes at a very low frequency.
