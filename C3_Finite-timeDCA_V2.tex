
\section{\label{sec:Finite-Time-Distributed-Consensu}Finite-Time Distributed
Consensus Algorithm}

Finite-Time Distributed Consensus Algorithm (FT-DCA) tries to find
the consensus value after each node run the FO-DCA algorithm iteratively
for a certain number of time. This is based on the assumption that
after a certain number of iterations, local values would contain sufficient
information to estimate the consensus value. (OK2.5)
\begin{defn}
Given the network $\mathcal{G}$, and initial local value vector $\mathbf{x}\left(0\right)$,
it is said the algorithm achieves a finite-time consensus if it solve
a consensus problem, and there exist a time $t^{*}$ and the consensus
value $\bar{x}$ such that $x_{i}\left(t\right)=\bar{x}$, for all
time $t>t^{*}$. 
\end{defn}
Motivated by the works in \cite{Kokiopoulou2007}, we try to decompose
the local value vector in a form which reveals an important property
of its iteration. Therefore we propose a filtering technique to estimate
the consensus value. (OK 3) 

In the network that adopts first order linear consensus algorithm,
each node has a number of consecutive local values obtained by Eq.\eqref{eq:1st iter. ni}.
There exists a linear filter. If each node passes its consecutive
local values through this filter, the output after a certain time
is the global average of the initial values over the network. Compared
to first order DCA algorithm, no more information exchanged between
nodes is needed. Such a filter is defined as the consensus finding
filter. (OK 2.5) 
\begin{defn}
\label{Def.A-consensus-finding}A consensus finding filter is a linear
filter defined by $\mathbf{h}\in R^{d}$, such that by passing through
local values $x_{i}(k)$ obtained by Eq.\eqref{eq:1st iter. ni},
the output $\hat{x}_{i}(k)=\mathbf{h}(k)*x_{i}(k)$ is the consensus
value $\overline{x}=\frac{1}{n}\sum_{i=1}^{n}x_{i}(0)$ of the network
after $k$ step $\left(k\geqslant m\right)$, where $m$ is the number
of distinct and nonzero eigenvalues of the weight matrix $\mathbf{W}$.
And $\mathbf{W}$ satisfies the convergence condition \ref{eq:convege condition W}.
\end{defn}

\subsection{\label{sub:Local-Value-Decomposition}Local Value Decomposition (OK
3)}

Suppose an undirected network with $n$ nodes, where duplex communication
is possible in each link. Therefore, the associated weight matrix
$\mathbf{W}\in\mathbf{R}^{n\times n}$ is symmetric and diagonalizable.
And there are $n$ linear independent eigenvectors of the weight matrix.
Thus, any initial value vector can be written in a linear combination
of these eigenvectors, 

\begin{equation}
\mathbf{x}\left(0\right)=\alpha_{1}\mathbf{e}_{1}+\alpha_{2}\mathbf{e}_{2}+\ldots+\alpha_{n}\mathbf{e}_{n}
\end{equation}
where $\alpha_{i}(i=1,2,\ldots,n)$ is the coefficient. For $k=1,2,3,...$
we have

\begin{eqnarray}
\mathbf{x}\left(k\right) & = & \mathbf{W}^{k}\left[\alpha_{1}\mathbf{e}_{1}+\alpha_{2}\mathbf{e}_{2}+\ldots+\alpha_{n}\mathbf{e}_{n}\right]\nonumber \\
 & = & \alpha_{1}\lambda_{1}^{k}\mathbf{e}_{1}+\alpha_{2}\lambda_{2}^{k}\mathbf{e}_{2}+\ldots+\alpha_{n}\lambda_{n}^{k}\mathbf{e}_{n}\label{eq:x(k) decomposition}\\
 & = & \sum_{i=1}^{n}\alpha_{i}\lambda_{i}^{k}\mathbf{e}_{i}\nonumber 
\end{eqnarray}
where $\lambda_{i}$ is the eigenvalue of $\mathbf{W}$. 

By rewriting $\mathbf{x}\left(k\right)$, it is clear that if all
eigenvalues of the weight matrix are known, the node values vector
is predictable and the consensus value can be found. For any node
$i=1,2,\ldots n$, its local value at time $k$ can be written as
\begin{equation}
x_{i}\left(k\right)=\alpha_{1}\lambda_{1}^{k}e_{i1}+\alpha_{2}\lambda_{2}^{k}e_{i2}+\ldots+\alpha_{n}\lambda_{n}^{k}e_{in}
\end{equation}
where $e_{ij}$ is the $i^{th}$ component of eigenvector $\mathbf{e}_{j}$.
Because of algebraic multiplicity of some eigenvalues, the equation
can be modified by combining the terms with the same eigenvalues.
Zero eigenvalues are ignored as they have no contribution in this
equation. Suppose $\mathbf{W}$ has $m$ distinct and nonzero eigenvalues,
denoted by $\lambda_{1},\lambda_{2},\ldots,\lambda_{m}$, then $x_{i}\left(k\right)$
evolves into 
\begin{equation}
x_{i}\left(k\right)=\beta_{i1}\lambda_{1}^{k}+\beta_{i2}\lambda_{2}^{k}+\ldots+\beta_{im}\lambda_{m}^{k}
\end{equation}
where $\beta_{ij}$ is the coefficient after combination. 


\subsection{\label{sub:Find-the-Consensus}Find the Consensus Value by Linear
Filter}

Let the sample vector $\mathbf{y}_{i}(k,d)\in R^{d}$ is defined by
the history of $x_{i}(k),$ 

\begin{equation}
\mathbf{y}_{i}(k,d)=\left[x_{i}(k),x_{i}(k-1),\ldots x_{i}(k-d+1)\right]^{\mathrm{T}}\label{eq:def. y(k,m)}
\end{equation}
and the Vandermonde matrix whose entries are the power of eigenvalues

\begin{equation}
\mathbf{V}(k,d)=\left[\begin{array}{cccc}
\lambda_{1}^{k} & \lambda_{2}^{k} & \cdots & \lambda_{m}^{k}\\
\lambda_{1}^{k-1} & \lambda_{2}^{k-1} & \cdots & \lambda_{m}^{k-1}\\
\vdots & \vdots & \ddots & \vdots\\
\lambda_{1}^{k-d+1} & \lambda_{2}^{k-d+1} & \cdots & \lambda_{m}^{k-d+1}
\end{array}\right]\label{eq:def. Lamda(k,m)}
\end{equation}
and $\mathbf{b}_{i}\left(d\right)=\left[\beta_{i1},\beta_{i2},\cdots\beta_{id}\right]^{\mathrm{T}}$,
then we have the following equation satisfied

\begin{equation}
\mathbf{y}_{i}(k,d)=\mathbf{V}(k,d)\mathbf{b}_{i}\left(d\right)\label{eq:Vander matrix and consensus-d*m}
\end{equation}
To obtain the consensus value $\bar{x}$ for node $i$, we need to
take sufficient samples of $x_{i}(k)$\textbf{ }and solve Eq.\eqref{eq:Vander matrix and consensus-d*m},
where $d$ should at least equal or larger than\textbf{ $m$}, which
is the number of distinct and nonzero eigenvalues of weight matrix
$\mathbf{W}$

Let $A(k,m)=\mathbf{V}^{-1}(k,m)$ and the first row of $A(k,m)$
is given by 
\begin{equation}
\mathbf{h}=\left[A_{11}(k,m),A_{12}(k,m),\ldots,A_{1m}(k,m)\right]^{\mathrm{T}}
\end{equation}
 we have 

\begin{equation}
\beta_{i1}=\mathbf{h}^{\mathrm{T}}\mathbf{y}_{i}(k,m)\label{eq:Find Consensus m order}
\end{equation}
 Eq.\eqref{eq:Find Consensus m order} shows that the consensus value
can be calculated by the filter defined in Def.\ref{Def.A-consensus-finding}. 

It is worth noting that Vandermonde matrix is related to a polynomial
interpolation problem and can be easily inverted in terms of Lagrange
basis polynomials \cite{Prass2007}. Due to this reason, this method
can be treated as an extrapolation method which find the consensus
value at infinity. At the same time, we only need to find out the
first coefficient $\beta_{i1}$ in the distributed averaging. Therefore,
only the elements in the corresponding row of $\mathbf{V}^{-1}(k,m)$
need to be found. This approach can save lots of computation time
in the inverting of Vandermonde matrix. 

Since simulation result shows that $\mathbf{h}$ is only depends the
associated Vandermonde matrix $\mathbf{V}(k,m)$ which is independent
of the nodes initial values and time index $k$. Therefore, each node
in the network could find the consensus value at any time $k\geqslant m$
by passing a number of consecutive local values through this filter.
In addition, all nodes in this network may share the same filter,
which means all the filters have the same impulse response. However,
such a consensus finding filter it is not unique. As an example, a
number of filters which have different impulse response or filter
lengths are found by different samples of $x_{i}(k)$.  Each node
could choose its own, but filter length determines the how many time-steps
before a node could find the consensus value. 

Suppose we take $d$ samples of $x_{i}(k)$, where $d>m$. Because
the Vandermonde matrix $\mathbf{V}(k,d)\in R^{d\times m}$ is non-square,
we introduce the Moore-Penrose pseudo inverse to find the least mean
square solution. 

Let

\begin{equation}
A(k,d)=\mathbf{V}^{+}(k,d)
\end{equation}
where $^{+}$ denote the Moore-Penrose pseudo inverse \cite{Piziak2007}.
And the first row of $A(k,d)$ is given by 

\begin{equation}
\mathbf{h}'=\left[A{}_{11}(k,d),A_{12}(k,d),\ldots,A_{1d}(k,d)\right]^{\mathrm{T}}
\end{equation}
Still, the value $\beta_{i1}=\mathbf{h}'^{\mathrm{T}}\mathbf{y}_{i}(k,m)$
is an accurate estimation of consensus value. Therefore, another consensus
finding filter is obtained.

Due to the multiplication of the consensus finding filter, the set
of filter is defined by

\begin{equation}
H=\{\mathbf{h}\in R^{d}|\forall d\geqslant m,\:\mathbf{h}^{\mathrm{T}}\mathbf{y}_{i}(k,d)=\bar{x}\}
\end{equation}
However, the shortest filter has its length equal to $m$, which means
node can only have the consensus value after $m$ steps. 

In \prettyref{sub:Numerical-Simulation}, the performance of this
algorithm is shown by comparing it with the first order DAC algorithm
using optimal weight matrix in \cite{Xiao2004}.


\subsubsection{Inverse the Vandermonde matrix}

The Vandermonde matrix has its application in some problems like polynomial
fitting, reconstruction of distributions from their moments, and so
on. Solving Vandermonde matrix is related to a polynomial interpolation
problem and can be easily inverted in terms of Lagrange basis polynomials.
It can be very difficult to invert in other way if the size of the
matrix is large, as the Vandermonde matrix is T is notoriously ill-conditioned
by its nature. It is a good way to always work on the problems related
to Vandermonde matrix in double precision or higher. 

Let $V_{m}$ be the Vandermonde's matrix of order $m$ given by 

\begin{equation}
V_{m}=\left[\begin{array}{cccc}
\lambda_{1} & \lambda_{2} & \cdots & \lambda_{m}\\
\lambda_{1}^{2} & \lambda_{2}^{2} & \cdots & \lambda_{m}^{2}\\
\vdots & \vdots & \ddots & \vdots\\
\lambda_{1}^{m} & \lambda_{2}^{m} & \cdots & \lambda_{m}^{m}
\end{array}\right]\label{eq:Vander Matrix}
\end{equation}
Then the equation 
\begin{equation}
\left[\begin{array}{cccc}
\lambda_{1} & \lambda_{2} & \cdots & \lambda_{m}\\
\lambda_{1}^{2} & \lambda_{2}^{2} & \cdots & \lambda_{m}^{2}\\
\vdots & \vdots & \ddots & \vdots\\
\lambda_{1}^{m} & \lambda_{2}^{m} & \cdots & \lambda_{m}^{m}
\end{array}\right]\left[\begin{array}{c}
\beta_{1}\\
\beta_{2}\\
\vdots\\
\beta_{m}
\end{array}\right]=\left[\begin{array}{c}
y_{1}\\
y_{2}\\
\vdots\\
y_{m}
\end{array}\right]\label{eq:Vander Eq.}
\end{equation}
is related to the problem of moments: Given the values of all $\lambda_{i}$,
find the unknown coefficients $b_{i}$, so that they match the given
values $y_{i}$ of the first $m$ moments. 

Its inverse is closely related to Lagrange's polynomial interpolation
formula. 

Let the polynomial of degree $m$ defined by 
\begin{equation}
P_{j}\left(\lambda\right)=\prod_{\begin{array}{c}
i=1\\
i\neq j
\end{array}}^{m}\frac{\lambda-\lambda_{i}}{\lambda_{j}-\lambda_{i}}=\sum_{k=1}^{m}b_{jk}\lambda^{k-1}\label{eq:Lagrange's polynomial}
\end{equation}
The polynomial $P_{j}\left(\lambda\right)$ a function of $\lambda$
and is specially designed so that it is equal to zero at all $\lambda_{i}$
except $i=j$ and takes on a value of one at $\lambda=\lambda_{j}$.
In other words, 
\[
P_{j}\left(\lambda_{i}\right)=\delta_{ij}=\sum_{k=1}^{m}b_{jk}\lambda_{i}^{k-1}
\]
where $\delta_{ij}=1$ when $i=j$ . The equation says that $b_{jk}$
is exactly the inverse of the matrix \eqref{eq:Vander Matrix}, with
the subscript $k$ as the column index. 

To drive the analytical expression of $b_{jk}$ and make it as easy
as possible, let's define some intermediate result. Define the polynomial
$q_{j}\left(\lambda\right)$ and work out its coefficients

\begin{eqnarray}
q_{j}\left(\lambda\right) & = & \frac{\prod_{i=1}^{m}\left(\lambda-\lambda_{i}\right)}{\left(\lambda-\lambda_{j}\right)}=\prod_{i=1,i\neq j}^{m}\left(\lambda-\lambda_{i}\right)\label{eq:intermindia polynomial}\\
 & = & c_{j,m}\lambda^{m-1}+c_{j,m-1}\lambda^{m-2}+\ldots+c_{j,2}\lambda+c_{j,1}\nonumber 
\end{eqnarray}
Examining the polynomial \ref{eq:Lagrange's polynomial} and polynomial
\ref{eq:intermindia polynomial}, we have 
\begin{eqnarray*}
b_{jk} & = & \frac{c_{j,m}}{q_{j}\left(\lambda_{j}\right)}
\end{eqnarray*}
Therefore, the solution of Eq.\ref{eq:Vander Eq.} is just the inverse
of Vandermonde matrix time the vector on the right. 
\[
\beta_{j}=\sum_{k=1}^{m}b_{jk}y_{k}
\]
If we only need to calculate the consensus value, as explained in
\ref{sub:Find-the-Consensus} only the elements in the corresponding
row of inverse Vandermonde's matrix need to be found. The computation
saving can be enormous by this approach.


\subsubsection{\label{sub:Numerical-Simulation}Numerical Simulation}

\begin{figure}
\hfill{}\includegraphics{\string"../Distributed Consensus Algorithm/Report_DAC/Net_Weight/graph with 8 nodes and 17 edges\string".pdf}\hfill{}\hfill{}\caption{\label{fig:Graph in Xiao'paper}Graph with optimal weights which maximize
convergence rate }
\end{figure}


Consider the graph from \cite{Xiao2004}, the weight matrix \textbf{$\mathbf{W}$}
corresponding to this graph is symmetric and has eigenvalues $\lambda(\mathbf{W})=\{1,0.6,0.4,0,0,0,-0.4,-0.6\}$.
The time index $k$ can be chosen large enough so that there are only
positive powers in the matrix $\mathbf{V}(k,d)$. For example, there
are 5 distinct and nonzero eigenvalues of \textbf{$\mathbf{W}$},
so we choose the time index $k=5$ and $d=5$ which is the minimum
filter length.

\[
\mathbf{V}(5,5)=\left[\begin{array}{ccccc}
1 & 0.0778 & 0.0102 & -0.0102 & -0.0778\\
1 & 0.1296 & 0.0256 & 0.0256 & 0.1296\\
1 & 0.216 & 0.064 & -0.064 & -0.216\\
1 & 0.36 & 0.16 & 0.16 & 0.36\\
1 & 0.6 & 0.4 & -0.4 & -0.6
\end{array}\right]
\]
 the first row of the inverse matrix $\mathbf{V}^{-1}(5,5)$ gives
the consensus finding filter 

\[
\mathbf{h}=\left[1.8601,\;0,\;-0.9673,\;0,\;0.1071\right]^{\mathrm{T}}
\]


\begin{figure}
\hfill{}\includegraphics[width=8cm]{\string"../Distributed Consensus Algorithm/Report_DAC/MSE/MSE_Filter vs FODAC\string".pdf}\hfill{}

\caption{\label{cap:perform. Consensus Filter}Performance of the first order
iteration with optimal matrix vs. consensus finding filter algorithm}
\end{figure}


For any random generated $\mathbf{x}(0)\in R^{n}$, node values vector
$\mathbf{x}(k)$ is updated by the iteration Eq.\eqref{eq:first order matrix}.
At the same time each node passes its local values though filter $\mathbf{h}$.
Filter output is given by $\hat{x}_{i}(k)=\mathbf{h}(k)*x_{i}(k)$.
Fig.\ref{cap:perform. Consensus Filter} compares the first order
DAC (FO-DAC) algorithm with optimal matrix and the proposed algorithm
with consensus finding filter. The performance is evaluated by the
mean square error (MSE), defined by $\mbox{MSE}_{\mbox{FO-DAC}}(k)=\sum_{i\in\mathcal{N}}E[\left|x_{i}(k)-\bar{x}\right|^{2}]$,
$\mbox{MSE}_{\mbox{filter}}(k)=\sum_{i\in\mathcal{N}}E[\left|\alpha_{i}(k)-\bar{x}\right|^{2}]$
respectively, where $\bar{x}=(1/n)\sum_{i\in\mathcal{N}}x_{i}(0)$.
The result shows that the consensus finding filter calculate the consensus
value after a finite number of iteration and MSE drops dramatically
to the quantization error at the same time. 


\subsection{Generalized FT-DAC: Unknown Network Topology (todo)}

In the section \ref{sub:Find-the-Consensus}, It shows if each node
has knowledge of the network topology (for example, eigenvalues of
the weight matrix), the consensus value can be calculated by a linear
combination of past local values. However, having knowledge of the
network topology in every node is a strong assumption. Although a
distributive algorithm to calculate the linear combination has also
been introduced in \cite{Sundaram2007}. This method requires several
runs of the consensus algorithm initialized with a set of linear independent
vectors. Alternatively, the original consensus algorithm can run many
instances in parallel with a set of independent initial values. However,
the data transmission at each iteration increases. In addition, this
method may not reliable to topology changes during the of multiple
re-initializations of original consensus algorithm. 

Here we propose a generalized finite-time consensus algorithm without
knowledge of network topology. It also does not require re-initialization
of the original consensus algorithm for several times. Before introducing
the algorithm, there are too important linear filter need to be introduced,
as the algorithm is based on the properties of these filters.


\subsubsection{Linear Predictor For Local Value (todo check 0)}

In observing the convergence behavior of each node's local value sequence,
one would come to the idea that the sequence must obeys some rule
as it converges. In section \ref{sub:Local-Value-Decomposition},
it is shown that local value vector can be decomposed in terms of
eigenvalues and eigenvectors. Based on this fact, a consensus estimation
method by inverting the Vandermonde matrix and an information flooding
techniques are proposed.

However, another important property of the FO-DCA need to be highlighted
in this section, because it will also inspire some ideas on signal
processing and applications in distributed systems.

To see how this property can be explained in an equation, we use the
concept of minimal polynomial of the weight matrix $\mathbf{W}$.
For any weight matrix $\mathbf{W}$, it has distinct eigenvalues denoted
by $\lambda_{1},\lambda_{2},\ldots,\lambda_{m},$ then the minimal
polynomial can be obtained by 
\[
p(\lambda)=\prod_{i=1}^{m}\left(\lambda-\lambda_{i}\right)^{r_{i}}
\]
where $r_{i}$ is the size of the largest Jordan block of $\mathbf{W}$
corresponding to eigenvalue $\lambda_{i}$. The minimal polynomial
can also be expanded into the form 
\[
p\left(\lambda\right)=\lambda^{r+1}+a_{r}\lambda^{r}+\ldots+a_{1}\lambda+a_{0}
\]
where $r+1$ is the degree of the minimal polynomial. Since the minimal
polynomial of matrix $\mathbf{W}$ satisfies $p\left(\mathbf{W}\right)=0$.
We have
\[
\mathbf{W}^{r+1}+a_{r}\mathbf{W}^{r}+\ldots+a_{1}\mathbf{W}+a_{0}\mathbf{I}=0
\]
if we multiply each side with initial value vector, and use the fact
$\mathbf{x}\left(k+1\right)=\mathbf{W}^{k+1}\mathbf{x}\left(0\right)$,
the above equation evolves into
\begin{equation}
\mathbf{x}\left(r+1\right)+a_{r}\mathbf{x}\left(r\right)+\ldots+a_{1}\mathbf{x}\left(1\right)+a_{0}\mathbf{x}\left(0\right)=0\label{eq:local value linear combination}
\end{equation}
Therefore, for any $k\geq r$
\begin{equation}
\mathbf{x}\left(k+1\right)=-a_{r}\mathbf{x}\left(k\right)-\ldots-a_{1}\mathbf{x}\left(k-r+1\right)+a_{0}\mathbf{x}\left(k-r\right)\label{eq:local value linear predictor}
\end{equation}
This equation shows that the vector sequence can be predict by a FIR
filter with coefficient given by $\left[-a_{r},-a_{r-1},\ldots,-a_{0}\right].$

Given the local value vector sequence obtained by FO-DCA, one may
instantly comes to the idea of applying an adaptive filter algorithm
to estimate the set of coefficient. For example, LMS, LSL and Kalman
filter algorithms. One advantage of the adaptive filter algorithm
is that when the network topology is changed, the filter could adaptively
change its coefficient during the iteration. 

In the next section \ref{sub:Cons.Find.Filter and relationship},
we will show that the relationship of linear predictor of local value
and the consensus finding filter.


\subsubsection{\label{sub:Cons.Find.Filter and relationship}The Consensus Finding
Filter and The Relationship with Linear Predictor (todo check)}

If matrix $\mathbf{W}$ satisfies the condition in \prettyref{thm:convergence condition}
there is a simple eigenvalue equals to one. As shown in section \ref{sub:Cons.Find.Filter and relationship},
the consensus finding filter is given by the row of inverse of Vandermonde's
matrix which corresponding to an eigenvalue equals to one. Without
loss of generality, let the first eigenvalue $\lambda_{1}=1$. Thus,
the corresponding row of the inverse of Vandermonde's Matrix is equal
to the consensus finding filter, whose coefficients are given by the
coefficients $b_{1k}$ in the polynomial. 

\begin{equation}
\mathbf{h}=\left[b_{11},b_{12},\ldots,b_{1m}\right]\label{eq:Consensus Finding filter Entry}
\end{equation}


Examining the minimal polynomial of weight matrix. (Since the symmetric
weight matrix has only $m$ distinct and non zero eigenvalues) 
\begin{equation}
p(\lambda)=\prod_{i=1}^{m}\left(\lambda-\lambda_{i}\right)=\lambda^{m}+a_{m}\lambda^{m-1}+\ldots+a_{2}\lambda+a_{1}\label{eq:Polynomial of matrix}
\end{equation}
and the Lagrange's polynomial interpolation formula \prettyref{eq:Lagrange's polynomial}
we can rewrite the $b_{jk}$ in terms of $a_{k}$. To make the expression
simple, we may also use the polynomial defined in \prettyref{eq:intermindia polynomial}.
Therefore, 
\begin{equation}
\mathbf{h}=\frac{\left[c_{1,m},c_{1,m-1},\ldots,c_{1,2}\right]}{\sum_{k=1}^{m}c_{1,k}}\label{eq:Consensus Find coeff. c}
\end{equation}
more specifically, we have $c_{1,k}=1+\sum_{j=k+1}^{m}a_{j}$. This
make it is possible for the adaptive filter to estimate the consensus
value by after all coefficients converge. 


\subsubsection{\label{sub:Finite-time-Consensus-on}Finite-time Consensus on generalized
condition}

Suppose the history of local values at node $x_{i}\left(k\right)$
is available at node $i$ after running one instance of FO-DCA algorithm.
Let the Toeplitz matrix be
\[
T=\left[\begin{array}{cccc}
x_{i}\left(k\right) & x_{i}\left(k-1\right) & \ldots & x_{i}\left(k-r\right)\\
x_{i}\left(k+1\right) & x_{i}\left(k\right) & \cdots\\
\vdots & \vdots & \ddots & \vdots\\
x_{i}\left(k+r-1\right) & x_{i}\left(k+r-2\right) & \cdots & x_{i}\left(k-1\right)\\
x_{i}\left(k+r\right) & x_{i}\left(k+r-1\right) & \cdots & x_{i}\left(k\right)
\end{array}\right]
\]
and $\mathbf{a}=\left[a_{r},\ldots,a_{1},a_{0}\right]^{T}$, $\mathbf{y}=\left[x\left(k+1\right),x\left(k+2\right),\ldots,x\left(k+r+1\right)\right]^{T}$.
Then they can be written in a matrix notation as

\begin{equation}
\mathbf{y}=T\mathbf{a}\label{eq:Toepliz Eq.}
\end{equation}
 Toeplitz matrix is a special type of matrix. It can be inverted by
some algorithm in the polynomial time of order $N^{2}$, rather than
the order of $N^{3}$ in general case (for example by LU decomposition).
This is an enormous computational saving. Levinson developed a recursive
algorithm to solve the system quickly if the Toeplitz matrix is symmetric.
However, the fact that the method can be generalized to the non-symmetric
case is not well known until it is stated in texts \cite{Robinson2000}.
Therefore, we can still invert this Toeplitz matrix using Levinson's
method. 

Now the requirement for node $i$ to solve Eq.\ref{eq:Toepliz Eq.}
is having sufficient local values. Once this is satisfied, node $i$
can solve the equation and obtain the set of coefficients \textbf{$\mathbf{a}$}
and construct the polynomial 
\[
p(\lambda)=\lambda^{r+1}+a_{r}\lambda^{r}+\ldots+a_{1}\lambda+a_{0}
\]
If the weight matrix have one simple eigenvalue equals to one, the
consensus finding filter can be easily obtain by defining another
polynomial 

\[
q\left(\lambda\right)=\frac{p\left(\lambda\right)}{\lambda-1}=c_{m}\lambda^{m-1}+c_{m-1}\lambda^{m-2}+\ldots+c_{2}\lambda+c_{1}
\]
and the consensus filter to be construct is given by the coefficient
in the new polynomial 
\[
\mathbf{h}=\frac{\left[c_{m},c_{m-1},\ldots,c_{2}\right]}{\sum_{k=1}^{m}c_{k}}
\]
where $c_{k}=1+\sum_{j=k+1}^{m}a_{j}$. 

It totally needs $2r+2$ local values and $2r+1$ iterations to construct
the Toeplitz matrix and Eq.\ref{eq:Toepliz Eq.}, but all the local
value can be obtained from one instance of FO-DCA. In contract, the
the original finite-time consensus algorithm in \cite{Sundaram2007}
requires $r$ iterations of FO-DCA for each instance, and totally
$r$ instances. Therefore, the proposed method has some improvement
to the original finite-time consensus algorithm. 

One interesting advantage of this method is that it can also apply
to the case when weight matrix is non-symmetric. It can be demonstrated
by just finding the inverse of a confluent Vandermonde matrix. In
examining the inverse of confluent Vandermonde matrix and find the
the expression of the row who corresponds to the unity eigenvalue,
we note that the difference is that the length of consensus filter.
It is not equal to the number of distinct and nonzero eigenvalues
$m$, but equal to the sum of all orders in minimal polynomial of
weight matrix $\sum_{i=1}^{m}r_{i}$. However, the relationship of
local value predictor and consensus finding filter still holds. Therefore,
the proposed method and doesn't need to change and generalizes to
the non-symmetric weight matrix case. 
