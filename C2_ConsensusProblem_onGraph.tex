
\chapter{\label{sec:Consensus-problem-on}Background of DAC}


\section{Prelimilary}

Consider a connected graph $\mathcal{G}=\left(\mathcal{V},\mathcal{E},{\cal A}\right)$
be a weighted digraph (or undirected graph) consisted by a set of
nodes $\mathcal{V}=\left\{ v_{1},v_{2},...,v_{n}\right\} $, a set
of edges $\mathcal{E}\subseteq\mathcal{V}\times\mathcal{V}$ and a
\textit{weighted adjacency matrix} ${\cal A}=\left[a_{ij}\right]$.
The edge $\left(v_{i},v_{j}\right)\in\mathfrak{\mathcal{E}}$ is a
ordered pair of distinct nodes, associated with an element in the
adjacency matrix ${\cal A}$, i.e. $\left(v_{i},v_{j}\right)\in\mathfrak{\mathcal{E}}\Leftrightarrow a_{ij}$.
For $v_{i}$, its set of communication neighbors is denoted by $\mathcal{N}_{i}=\text{ }\left\{ v_{j}\in{\cal V}|\left(v_{i},v_{j}\right)\in\mathcal{E}\right\} $.
And $\left|\mathcal{N}_{i}\right|$ denotes the number of neighbors
of $v_{i}$. We assume $a_{ij}=0$, if $v_{j}\notin{\cal N}_{i}$
(note that $a_{ii}=0$, as $v_{i}\notin{\cal N}_{i}$) for all $v_{i}\in\mathcal{V}$.
Any node $v_{i}$ can only transmit information to other nodes that
belong to the neighbors set. 

For each edge $e_{k}=\left(v_{i},v_{j}\right)\in{\cal E}$, we arbitrarily
choose one end of $e_{k}$ to be positive and another to be negative.
The\textit{ incidence matrix} $Q=Q\left({\cal G}\right)\in\mathbb{R}^{n\times\left|{\cal E}\right|}$
is defined by 
\[
Q=\left(q_{ik}\right)=\begin{cases}
1 & \mbox{if }v_{i}\mbox{ is the positive end of }e_{k}\\
-1 & \mbox{if }v_{i}\mbox{ is the negative end of }e_{k}\\
0 & \mbox{otherwise}
\end{cases}
\]
The \textit{Laplacian matrix} of the graph ${\cal G}$ is defined
by $L=L\left({\cal G}\right)=QQ^{\mathrm{T}}$. In fact $L$ is independent
of the choice of positive ends of the edges. It is a symmetric, positive
semidefinite, singular matrix \cite{Russell1994}.

The \textit{Laplacian spectrum} of a graph \cite{Das2004} is denoted
by 
\[
S\left(L\right)=\left\{ \lambda_{1}\left(L\right),\lambda_{2}\left(L\right),\ldots,\lambda_{n}\left(L\right)\right\} 
\]
where $\lambda_{i}\left(L\right)$ is the eigenvalue of $L$ . When
refer to multiple matrix, we use $\lambda_{i}\left(A\right)$ to denote
the eigenvalues of matrix $A$. We assume these eigenvalues are ordered
non-decreasingly so that, $0=\lambda_{1}\left(L\right)\leq\lambda_{2}\left(L\right)\leq\ldots\leq\lambda_{n}\left(L\right)$.
The second smallest eigenvalue $\lambda_{2}\left(L\right)$ is called
algebraic connectivity of the graph. 

Suppose each node holds an initial scalar local value $x_{i}\left(0\right)\in\mathbb{R}$,
which can be a value that locally acquired by node representing physical
quantities such as temperature, humidity, illumination, attitude,
etc. We define the \textit{local value vector} $\mathbf{x}\left(0\right)=\left[x_{1}\left(0\right),...,x_{n}\left(0\right)\right]^{T}\in\mathbb{R}^{n}$
to represent all the initial values on the network. The network is
said to be reached a consensus if and only if $x_{i}=x_{j}$, for
all nodes $v_{i},v_{j}\in{\cal V},i\neq j$. In the other words, all
the nodes are in an agreement of a quantity, the common value of all
nodes is called the \textit{consensus value}.


\subsection{Consensus problem on Graphs}

To describe the behavior of each node or agent, suppose each node
has the following dynamics 

\begin{equation}
\dot{x}_{i}=f\left(x_{i},u_{i}\right),i\in{\cal V}
\end{equation}
and the graph (or network) is a system has the dynamics

\begin{equation}
\mathbf{\dot{x}}=F\left(\mathbf{x},\mathbf{u}\right)\label{eq:system dynamic}
\end{equation}
where $F\left(\mathbf{x},\mathbf{u}\right)$ is the columnwise concatenation
of individual dynamics $f\left(x_{i},u_{i}\right)$, for all node
$i=1,\ldots,n$. In an ad-hoc network with the mobile nodes, the topology
$G$\textbf{ }is switching and the system will updating its $F\left(\mathbf{x},\mathbf{u}\right)$
from time to time. 

The input or feedback $u_{i}$ in the node's dynamic is a function
of the historical states of node $i$ and its neighbors
\begin{equation}
u_{i}=g\left(x_{j_{1}},x_{j_{2}},\ldots,x_{j_{m_{i}}}\right)\label{eq:consensus protocol}
\end{equation}
where $j_{1},\ldots,j_{m_{i}}$ are the node indexes that belong to
the set $\left\{ i\right\} \cup{\cal N}_{i}$. \prettyref{eq:consensus protocol}
is called a consensus protocol under topology $G$. If the network
graph is not fully connected, it is said to be a distributed consensus
protocol. 
\begin{defn}
Let ${\cal X}:R^{n}\to R$ be a function of $n$ variables of $x_{1},x_{2},\ldots,x_{n}$
and let $\mathbf{x}\left(0\right)$ denotes the initial condition
of the network. The ${\cal X}$-consensus problem is a distributed
method to calculate the consensus value ${\cal X}\left(\mathbf{x}\left(0\right)\right)$
in a graph $G$. 
\end{defn}
Consensus problems can be different by their consensus values. For
instance, we give the definition of the average consensus ${\cal X}\left(\mathbf{x}\right)=\frac{1}{n}\sum_{i=1}^{n}x_{i}\left(0\right)$,
maximum consensus ${\cal X}\left(\mathbf{x}\right)=\max\left(\mathbf{x}\right)$,
minimum consensus ${\cal X}\left(\mathbf{x}\right)=\min\left(\mathbf{x}\right)$
and variance consensus ${\cal X}\left(\mathbf{x}\right)=\mbox{var}\left(\mathbf{x}\right)$
by their expressions respectively. The average consensus is a special
case of consensus problem, which computing the average of all initial
values $\overline{x}=\mbox{mean}\left(\mathbf{x}\right)=\frac{1}{n}\sum_{i=1}^{n}x_{i}\left(0\right)$
using a distributed system dynamics $\mathbf{\dot{x}}=F\left(\mathbf{x},\mathbf{u}\right)$
in a network $G$. 

We are interested in the distributed solutions of the consensus problem
as the network only allows an node to communicate with its neighbors.
We say the protocol \prettyref{eq:consensus protocol} solves the
consensus problem asymptotically if and only if there exists an asymptotically
stable state $x^{*}={\cal X}\left(\mathbf{x}\right)$ of system dynamics
\prettyref{eq:system dynamic}, which satisfying for all $\delta>0$,
there exists a time index $t^{*}>0$, such that $\left|x_{i}(t)-x^{*}\right|<\delta$
for all $t>t^{*}$ and $\forall i\in{\cal V}$.

Maximum or minimum consensus is very similar to information flooding.
Each node $v_{i}$ in the network just compare the local values held
by itself and others in ${\cal N}_{i}$, then broadcast the maximal
or minimal values. Therefore, maximum/minimum consensus can be finished
in a number of step that equal to the diameter of the graph. For upper
boundary of the diameter, see \cite{Russell1994}.

However, average consensus is more challenging problem than the maximum
(or minimum) consensus. Because, the average value is a linear combination
of all the initial states of network nodes and the condition $x_{i}^{*}={\cal X}\left(\mathbf{x}\right)$
for all $i$ has to be satisfied. In this chapter, we will discuss
the average consensus and especially its optimization problem, which
is solved distributively by discrete-time matrix iteration. Furthermore,
the variance consensus problem can be solved by two instances of average
consensus, because we have the relation $\mbox{var}\left(\mathbf{x}\right)=\mbox{mean}\left(\mathbf{x}\cdot\mathbf{x}\right)-\left[\mbox{mean}\left(\mathbf{x}\right)\right]^{2}$.
Thus, in the following of this work we focus on the average consensus
problem and the distributed average consensus (DAC) algorithms. 


\subsection{Continuous-time vs. Discrete-time Consensus}

In this section, we will show the difference between continuous-time
and discrete-time consensus 

The dynamics of the consensus protocol is given by 

\begin{equation}
\dot{x}_{i}\left(t\right)=u_{i}\left(t\right)\label{eq:continuous_time_consensus}
\end{equation}
where $u_{i}\left(t\right)$ is a consensus protocol 
\begin{equation}
u_{i}\left(t\right)=\sum_{j\in{\cal N}_{i}}a_{ij}\left(x_{j}-x_{i}\right)\label{eq:1st Order Protocol}
\end{equation}
that solves the average consensus problem ${\cal X}\left(\mathbf{x}\right)=\mbox{mean}\left(\mathbf{x}\right)$. 

The continuous-time consensus requires the nodes in a network have
dynamics in a form of differential equations. Continuous-time consensus
involves analog signals that are easily interfered by channel noise.
Consequently, the finally consensus value will be a random variable
with mean equals to the consensus value without noise and variance
proportional to the signal noise ratio. It is shown in \cite{Kar2009},
the variance is also proportional to time $t$ hence it is increasing
as the algorithm executing. 



On the other hand, discrete-time consensus only involves the quantization
error during the algorithm execution as long as the data packages
are correctly received.  Nowadays, sensor nodes with digital processing
unit becomes more cheaper, we can get more benefits from the advantages
of discrete-time consensus algorithms. Consequently, discrete-time
consensus algorithms are mainly discussed in the following of this
thesis. 


\subsubsection{Continuous-time Consensus}

The continuous-time consensus with system dynamics \prettyref{eq:continuous_time_consensus}
is a linear differential equation 
\begin{equation}
\mathbf{\dot{x}}\left(t\right)=-{\cal L}\mathbf{x}\left(t\right)\label{eq:system differential dynamics}
\end{equation}
Solving the differential equation \prettyref{eq:system differential dynamics}
will yield a continuous-time solution in an exponential matrices form
\begin{equation}
\mathbf{x}\left(t\right)=\exp\left(-{\cal L}t\right)\mathbf{x}\left(0\right)\label{eq:x(t) of CT-1st-DAC}
\end{equation}
where ${\cal L}$ is called the \textit{weighted graph Laplacian}
associated with network graph ${\cal G}$, which is defined by

\begin{equation}
l_{ij}=\begin{cases}
\sum_{k=1,k\neq i}^{n}a_{ik} & j=i\\
-a_{ij} & j\neq i
\end{cases}\label{eq:Graph Laplacian def.}
\end{equation}
For a graph with 0-1 adjacency, the weighted graph Laplacian can be
denoted in another form, which is unweighted Laplacian matrix denoted
by $L$

\begin{equation}
l_{ij}=\begin{cases}
\left|{\cal N}_{i}\right| & j=i\\
-1 & j\in{\cal N}_{i}\\
0 & \mbox{otherwise}
\end{cases}\label{eq:Laplacian def.}
\end{equation}
In some literature \cite{Xiong2009a}, they use the second definition
(\prettyref{eq:Laplacian def.}) of the Laplacian matrix to analyze
the convergence rate of DAC algorithm. It is a special case when weights
$a_{ij}$ for all edges in ${\cal E}$ equal to one. Therefore, to
distinguish them, we denote the weighted graph Laplacian matrix and
the unweighted Laplacian matrix by ${\cal L}$ and $L$ respectively. 


\paragraph*{Convergence Conditions for Continuous-time Consensus}

There is some important theories for continuous-time consensus convergence
problem. To provides the necessary and sufficient conditions of the
graph Laplacian matrix so that a convergent consensus algorithm could
be carried out on the network , some result is induced from Perron-Frobenius
theorem and Gerschgorin's theorem that gives the upper and lower boundary
of the spectral radius. 


\begin{thm}
\cite{Olfati-Saber2004} \label{thm:Bounds of eig(L)-1}Let the graph
has the Laplacian matrix ${\cal L},$ denote the maximum node outdegree
of the graph by 
\begin{equation}
d_{max}=\max_{1\leq i\leq n}\left(\sum_{j=1,j\neq i}^{n}l_{i,j}\right)
\end{equation}
Then, all the eigenvalues of ${\cal L}$ are located in the following
disk,
\begin{equation}
\left|z-d_{max}\right|\leq d_{max}
\end{equation}
which is centered at $z=d_{max}+0j$ on the complex plane.\end{thm}
\begin{proof}
Based on the Gerschgorin's theorem, all the eigenvalue of ${\cal L}$
are located in the union of the disks.
\begin{equation}
\left|z-l_{i,i}\right|\leq\sum_{j=1,j\neq i}^{n}\left|l_{i,j}\right|,\ 1\leq i\leq n.\label{eq:disk Union of eig(L)-1}
\end{equation}
Since ${\cal A}=\left[a_{i,j}\right]$ is an non-negative matrix,
by the definition of Laplacian matrix, $l_{i,j}\leq0$ and $l_{i,i}=\sum_{k=1,k\neq i}^{n}a_{ik}\geq0.$
Therefore, let $d_{i}=l_{i,i}$ and 
\begin{equation}
d_{i}=\sum_{j=1,j\neq i}^{n}\left|l_{i,j}\right|
\end{equation}
 and the union of disks becomes
\begin{equation}
\bigcup_{1\leq i\leq n}\left\{ z\in\mathbb{C}:\left|z-d_{i}\right|\leq d_{i}\right\} 
\end{equation}
On the other hand, all these disks are located inside the largest
disk with radius $d_{max}$. This result ends the proof of the theorem
\end{proof}
Based on the \prettyref{thm:Bounds of eig(L)-1}, it is obvious all
the nonzero eigenvalues of ${\cal L}$ have positive real parts. This
immediately lead to a convergence theorem of the continuous-time consensus
protocol \prettyref{eq:1st Order Protocol}. Since all the nonzero
eigenvalue of $-{\cal L}$ located in the disk $\left|z+d_{max}\right|\leq d_{max}$,
and the eigenspace associated with zero is one-dimensional, the eigenvector
associated with zero eigenvalue has the form $\alpha\mathbf{1}$,
i.e. $x_{i}=\alpha$ for all $i$. This result will be very helpful
as the negative real part can guarantee that the system dynamic is
stable and convergent, see figure \ref{fig:Boundary-of-eigenvalues-1}.

We will prove that the solution given in exponential matrices form
\prettyref{eq:x(t) of CT-1st-DAC} converge to a consensus value as
$t\to\infty$ in the next. 

\begin{figure}
\hfill{}\includegraphics{\string"Graph/location of laplacian eigenvalue\string".pdf}\hfill{}\hfill{}

\caption{\label{fig:Boundary-of-eigenvalues-1}Boundary of eigenvalues of Laplacian
matrix}
\end{figure}




Considering the system dynamic $x\left(t\right)=\exp\left(-{\cal L}t\right)x\left(0\right)$.
Because $\exp\left(-{\cal L}t\right)$ is a non-negative matrix, the
Perron-Frobenius theorem tell us that $\exp\left(-{\cal L}t\right)$
has a positive real eigenvalue equals to one which is also the spectral
radius. Together with the \prettyref{thm:Bounds of eig(L)-1} which
implies that all the eigenvalues of $-{\cal L}$ have negative real
part, see figure \ref{fig:Boundary-of-eigenvalues-1}, we immediately
come to the following theorem.
\begin{thm}
Assume ${\cal G}=\left({\cal V},{\cal E},{\cal A}\right)$ is strongly
connected graph, and the associated graph Laplacian matrix ${\cal L}$
is defined in \prettyref{eq:Graph Laplacian def.}, which has only
one zero eigenvalue. Let the $u_{r}$ is the uniformed right eigenvector
and $u_{l}$ is the uniformed left eigenvector associated with the
zero eigenvalue, i.e. $Lu_{r}=0$, $u_{l}^{T}L=0$, then we have $u_{l}^{T}u_{r}=1$
and the system dynamic 
\begin{equation}
x\left(t\right)=\exp\left(-{\cal L}t\right)x\left(0\right)
\end{equation}
will have the stable state of the system dynamic given by

\begin{equation}
x\left(t\right)=Kx\left(0\right)
\end{equation}
 where K is a matrix in $R^{n}$, and $K=\lim_{t\to\infty}\exp\left(-{\cal L}t\right)=u_{r}u_{l}^{T}$. \end{thm}
\begin{proof}
Let $A=-{\cal L}$, and it has a Jordan form of $A=UJU^{-1}$. Then
we can have $\exp\left(At\right)=U\exp\left(Jt\right)U^{-1}$. Because
$A$ has all its eigenvalue except a simple zero eigenvalue have negative
real part, thus, as $t\to\infty$, all other Jordan block vanish,
and $\exp\left(Jt\right)$ converges to a matrix with only single
nonzero entry, denoted by $Q$. Since matrix $U$ contains a column
which associated with the zero eigenvalue of $A$ is $u_{r}$, similarly,
$U^{-1}$ has a corresponding row equals to $u_{l}$. By simply calculating
the equation $K=UQU^{-1}$, we can show that $K=u_{r}u_{l}^{T}$.
And the fact $U^{-1}U=I$ shows that $u_{l}^{T}u_{r}=1$ . This ends
the proof. 
\end{proof}
For the average consensus problem, it is obvious that all the elements
in $K$ must equal to $\frac{1}{n}$. This requires the graph Laplacian
${\cal L}$ satisfies the conditions: ${\cal L}\mathbf{1}=0$, $\mathbf{1}^{T}{\cal L}=0$,
where $\mathbf{1}\in R^{n}$ is an all unity vector. And the $u_{r}$
and $u_{l}$ will change into vectors with equivalent constant in
all its components. If they are uniformed, then $u_{r}=u_{l}=\frac{1}{\sqrt{n}}.$ 


\subsubsection{Discrete-time Consensus}

The discrete-time consensus with the dynamics Eq.\prettyref{eq:continuous_time_consensus}
is 
\begin{equation}
x_{i}\left(k+1\right)=x_{i}\left(k\right)+u_{i}\left(k\right)\label{eq:discre-time consensus-1}
\end{equation}
For network agents have discrete-time consensus protocol, their system
dynamics can be given in a matrix form
\begin{equation}
\mathbf{x}(k+1)=W\mathbf{x}(k)
\end{equation}
where $W=\left[w_{ij}\right]=I-{\cal L}$. We say the iteration is
convergent if there exists a vector denoted by $\mathbf{x}^{*}$,
which satisfies.

\begin{equation}
\mathbf{x}^{*}=W\mathbf{x}^{*}\label{eq:stable state iteration}
\end{equation}
Moreover, 

\begin{equation}
\mathbf{x}(k+1)-\mathbf{x}^{*}=W\left[\mathbf{x}(k)-\mathbf{x}^{*}\right]\label{eq:error vector iter. 1st}
\end{equation}
\prettyref{eq:stable state iteration} states that $\mathbf{x}^{*}$
is a right eigenvector of matrix $W$ associated with a simple eigenvalue
1. For  convergence conditions and more details about discrete-time
first-order DAC algorithm, see \prettyref{sub:DT-First-Order-DAC}. 


\section{Categories of Distributed Consensus Average Algorithm}

Distributed average consensus (DAC) algorithms is one of the most
widely used consensus algorithm.

To introduce the DAC algorithm, we may started from one of the simplest
one in their family. As shown in \prettyref{sub:DT-First-Order-DAC}.
(\prettyref{fig:Categories-of-Discrete-time} only shows the branch
of discrete-time distributed consensus algorithm. Although continuous-time
consensus algorithms have some similarities with discrete-time ones,
their performance are quite different when Gaussian white noise exists).
The first order DAC algorithm is the simplest algorithm that can solve
the consensus problem in a number of iteration. It's convergence rate
is related to the spectral radius of a graph dependent matrix. So
the optimization problem is to find the optimal matrix with minimum
spectral radius. However, global information of the graph matrix must
be available. In distributed methods, this is a quite strong condition.
Without the global information, best constant and metropolitan matrix
can be the sub-optimal solution to the consensus problem \cite{Xiao2004}.
The first order DAC algorithm is one of the asymptotic algorithms
together with higher-order DAC algorithm. The demand of higher-order
algorithm is mainly because the requirement for a fast convergence
rate is always a matter of issue. In \prettyref{sub:Discrete-High-Order},
we introduced higher-order DAC algorithm, which could have faster
convergence rate, and no additional requirement and communication
cost is required compared with first-order DAC \cite{Xiong2010}.
Therefore, it is shortly applied into practical consensus problem
after invented.  

Some of the novel methods can solve the average consensus problem
in finite number of iterations. These methods are referred to as finite-time
consensus algorithm \cite{Sundaram2007}. It actually a very sophisticated
signal processing technique that find the asymptotically stable equilibrium
$\bar{x}$ by extrapolation, see \prettyref{sec:Finite-Time-Distributed-Consensu}.
A derivative of finite-time consensus algorithm is the adaptive filter
algorithm for average consensus . It applies an adaptive filter algorithm
to asymptotically converge the set of parameters which are required
in the computation of consensus value \cite{Cavalcante2010}. As a
contribution of this thesis, a method to calculate these necessary
parameters by inverting Toeplitz matrix is introduced in \prettyref{sub:Finite-time-Consensus-on}. 

\begin{figure}
\hfill{}\includegraphics[width=13.5cm]{\string"Graph/Categories of DAC_Nor\string".pdf}\hfill{}

\caption{\label{fig:Categories-of-Discrete-time}Categories of Discrete-time
Consensus Algorithm}
\end{figure}



\subsection*{}
